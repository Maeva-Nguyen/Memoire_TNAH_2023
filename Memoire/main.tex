\documentclass[a4paper,12pt,twoside]{book}
\usepackage[T1]{fontenc}
\usepackage{inputenc}
\usepackage{fontspec}
\usepackage{lmodern}
\usepackage[english,french]{babel}

% Paquets pour le chinois
\usepackage{xeCJK}
\setCJKmainfont{Noto Sans CJK SC}
\usepackage[overlap,CJK]{ruby}
\usepackage{xpinyin}

\usepackage{xspace} % pour la gestion des espaces après les commandes
\usepackage{minted} % colored source code
\usepackage{csquotes}

% Mise en page École des chartes
\usepackage[margin=2.5cm]{geometry} % marges
\usepackage{setspace}
\onehalfspacing % interligne de 1.5
\setlength\parindent{1cm}

\usepackage{graphicx}

\usepackage[backend=biber, sorting=nyt, style=enc, minbibnames=10, maxbibnames=10]{biblatex}
\addbibresource{bibliographie/bibliographie.bib}
\nocite{*}
\defbibnote{intro}{Cette bibliographie présente toutes les ressources utilisées, de tout type, citées ou non, par simple ordre alphabétique.}


\usepackage[pdfusetitle, pdfsubject={Mémoire TNAH — Titre}, pdfkeywords={mot1, mot2, mot3}]{hyperref}

\author{Maëva Nguyen – M2 TNAH — ENC}
\title{Titre mémoire}

% ACRONYMS
\usepackage[automake, acronym, toc]{glossaries}
\makeglossaries
\setacronymstyle{short-long}
\newacronym{fair}{\textsc{FAIR}}{\emph{Findable Accessible Interoperable Reusable}}
\newacronym{COREL}{\textsc{COREL}}{\emph{Code relationnel}}
\newacronym{EFEO}{\textsc{EFEO}}{\emph{École Française d'Extrême-Orient}}
\newacronym{IHEC}{\textsc{IHEC}}{\emph{Institut des hautes études chinoises}}
\newacronym{CollEx}{\textsc{CollEx}}{\emph{Collections d'excellence}}
\newacronym{EPHE}{\textsc{EPHE}}{\emph{École pratique des hautes études}}
\newacronym{EHESS}{\textsc{EHESS}}{\emph{École des hautes études en sciences sociales}}
\newacronym{IR}{\textsc{IR}}{\emph{Infrastructure de recherche}}
\newacronym{LSC}{\textsc{LSC}}{\emph{Legalizing space in China}}
\newacronym{EPJ}{\textsc{EPJ}}{\emph{Emerging procedural justice}}
\newacronym{POC}{\textsc{POC}}{Proof of concept}
\newacronym{ANR}{\textsc{ANR}}{\emph{Agence nationale de la recherche}}
\newacronym{XML}{\textsc{XML}}{\emph{Extensible Markup Language}}
\newacronym{IIIF}{\textsc{IIIF}}{\emph{International image interoperability framework}}
\newacronym{JSON}{\textsc{JSON}}{\emph{JavaScript Object Notation}}
\newacronym{SQL}{\textsc{SQL}}{\emph{Structured Query Language}}
\newacronym{SGBD}{\textsc{SGBD}}{\emph{Système de gestion de base de données}}
\newacronym{IDE}{\textsc{IDE}}{\emph{Integrated development environment}}
\newacronym{FTP}{\textsc{FTP}}{\emph{File Transfer Protocole}}
\newacronym{HTML}{\textsc{HTML}}{\emph{HyperText Markup Language}}
\newacronym{ODD}{\textsc{ODD}}{\emph{One Document Does It All}}
\newacronym{TEI}{\textsc{TEI}}{\emph{Text Encoding Initiative}}
\newacronym{DTD}{\textsc{DTD}}{\emph{Document Type Definition}}
\newacronym{OCR}{\textsc{OCR}}{\emph{Optical Character Recognition}}

% COMMANDS
\newcommand{\enc}{École nationale des chartes\xspace}
\newcommand{\fair}{\gls{fair}\xspace}
\newcommand{\api}{\gls{api}\xspace}
\newcommand{\XIV}{\textsc{xiv}\ieme{}\xspace}
\newcommand{\COREL}{\gls{COREL}\xspace}
\newcommand{\cdf}{Collège de France\xspace}
\newcommand{\EFEO}{\gls{EFEO}\xspace}
\newcommand{\IHEC}{\gls{IHEC}\xspace}
\newcommand{\CollEx}{\gls{CollEx}\xspace}
\newcommand{\EPHE}{\gls{EPHE}\xspace}
\newcommand{\EHESS}{\gls{EHESS}\xspace}
\newcommand{\IR}{\gls{IR}\xspace}
\newcommand{\LSC}{\gls{LSC}\xspace}
\newcommand{\EPJ}{\gls{EPJ}\xspace}
\newcommand{\code}{\og code virtuel \fg \xspace}
\newcommand{\POC}{\gls{POC}\xspace}
\newcommand{\ANR}{\gls{ANR}\xspace}
\newcommand{\XML}{\gls{XML}\xspace}
\newcommand{\IIIF}{\gls{IIIF}\xspace}
\newcommand{\JSON}{\gls{JSON}\xspace}
\newcommand{\genyuan}{\textit{Da Qing lüli genyuan}\xspace}
\newcommand{\huidian}{\textit{Huidian shili}\xspace}
\newcommand{\dc}{\textit{Duli cunyi}\xspace}
\newcommand{\dq}{\textit{Da Qing lüli}\xspace}
\newcommand{\dqlj}{\textit{Da Qing lü jijie fuli\xspace}}
\newcommand{\SQL}{\gls{SQL}\xspace}
\newcommand{\SGBD}{\gls{SGBD}\xspace}
\newcommand{\IDE}{\gls{IDE}\xspace}
\newcommand{\FTP}{\gls{FTP}\xspace}
\newcommand{\HTML}{\gls{HTML}\xspace}
\newcommand{\ODD}{\gls{ODD}\xspace}
\newcommand{\TEI}{\gls{TEI}\xspace}
\newcommand{\lu}{\textit{lü}\xspace}
\newcommand{\li}{\textit{tiaoli}\xspace}
\newcommand{\DTD}{\gls{DTD}\xspace}
\newcommand{\cv}{\og code virtuel \fg}
\newcommand{\tp}{\textit{TEI Publisher}\xspace}
\newcommand{\OCR}{\gls{OCR}\xspace}

% Pour retirer le titre courant d'une page vide avant un chapitre
\newcommand{\clearemptydoublepage}{\newpage{\pagestyle{empty}\cleardoublepage}}
% Pour des sections non numérotées dans la table des matière
\newcommand\chapterNo[1]{
  \chapter*{#1}
  \markright{\MakeUppercase{#1}}
}

\begin{document}

\onehalfspacing 

\frontmatter

    \begin{titlepage}
	\begin{center}
		
		\bigskip
		
		\begin{large}
			ÉCOLE NATIONALE DES CHARTES
		\end{large}
		\begin{center}\rule{2cm}{0.02cm}\end{center}
		
		\bigskip
		\bigskip
		\bigskip
		\begin{Large}
			\textbf{Maëva Nguyen}\\
		\end{Large}
		\begin{normalsize} \textit{licenciée ès lettres}\\
		\end{normalsize}
		
		\bigskip
		\bigskip
		\bigskip
		
		\begin{Huge}
			\textbf{Titre}\\
		\end{Huge}
		\bigskip
		\bigskip
		\begin{LARGE}
			\textbf{Sous-titre}\\
		\end{LARGE}
		
		\bigskip
		\bigskip
		\bigskip
		\begin{large}
		\end{large}
		\vfill
		
		\begin{large}
			Mémoire 
			pour le diplôme de master \\
			\og Technologies numériques appliquées à l'histoire~\fg\\
			\bigskip
			2023
		\end{large}
		
	\end{center}
\end{titlepage}

    \thispagestyle{empty}	
    \cleardoublepage
	
    \chapterNo{Résumé}
\addcontentsline{toc}{chapter}{Résumé}
\medskip	

Résumé\\

\textbf{Mots-clés~:} mot1~; mot2~; mot3~.\\

\textbf{Informations bibliographiques~:} Maëva Nguyen, \textit{Titre du mémoire}, mémoire de master \og Technologies numériques appliquées à l'histoire~\fg, dir. Ségolène Albouy, École nationale des chartes, 2023.

\clearemptydoublepage
	
    \chapterNo{Remerciements}
    \addcontentsline{toc}{chapter}{Remerciements}
    
    \chapterNo{Introduction}
    \addcontentsline{toc}{chapter}{Introduction}

    \thispagestyle{empty}
    \cleardoublepage

\mainmatter

    \part{Le projet COREL, un projet de recherche inter-institutionnel}
        \chapter{Le projet COREL}
        Le projet \COREL est un projet de recherche inter-institutionnel entre le \cdf et l'\EFEO. Ce projet pluridisciplinaire réunit les humanités numériques et l'histoire du droit chinois et initie ainsi la collaboration entre historiens du droit, sinologues et ingénieurs en sciences humaines.
                    \section{La collaboration entre les institutions}
\subsection{Le Collège de France}

\subsubsection{Historique}
Le \cdf naît sous le règne de François I\ier{} lorsqu'en 1520, Guillaume Budé, libraire du roi, demande la création d'une institution regroupant des professeurs. C'est ainsi que sont nommés trois professeurs d'hébreux, deux de grec et un de mathématiques en 1530 par François I\ier{} : les lecteurs royaux. La création du \cdf est issue de la croissance de l'humanisme et, en accord avec sa devise \textit{\og Docet omnia \fg}, il enseigne \og tout \fg. C'est d'abord une institution "bâti[e] en hommes", comme le dit Pierre Bayle, qui ne possède pas de siège. Cependant, les cours ont essentiellement lieu place de Cambrai, et un bâtiment y est construit à la fin du XVIII\ieme siècle. Aujourd'hui encore, les cours donnés au \cdf sont publics et ouverts à tous. L'institution s'inscrit ainsi dans une tradition de \og science ouverte \fg. 

\subsubsection{Organisation actuelle}
Le \cdf est divisé en plusieurs chaires. Les chaires statutaires sont attribuées à des professeurs élus, qui occupent cette position jusqu'à l'âge de 73 ans. Les chaires annuelles et internationales sont occupées par des chercheurs choisis par l'Assemblée du \cdf pour des périodes plus courtes. En plus des cours ouverts à tous publics et donnés sur place, le \cdf met à disposition des cours en ligne, des enregistrements, des podcasts, etc. afin de diffuser le savoir plus librement via leur site internet et une chaîne YouTube. 

\subsubsection{La bibliothèque d'études chinoises}
La Bibliothèque d'études chinoise du \cdf a été fondée en 1927 par Paul Pelliot et Marcel Granet. Elle est rattachée à l'\IHEC, lui-même fondé en 1920. La bibliothèque de l'\IHEC rejoint le \cdf en 1972. Elle possède un fonds de 120 000 oeuvres, dont 1 600 périodiques, principalement de la Chine impériale et pré-impériale. Cette collection a reçu le label \CollEx en 2021. La bibliothèque met à la disposition du projet \COREL des sources juridiques et leurs numérisations. Elle est le porteur administratif du projet.

\subsection{L’École Française d'Extrême-Orient}

\subsubsection{Historique}
L'\EFEO a été créée en 1898 à Hô Chi Minh-Ville (aussi appelée Saigon) sous le nom de \og Mission archéologique d'Indo-Chine \fg. L'institution naît sous l'influence du courant orientaliste du XIX\ieme siècle en France, afin d'encourager les chercheurs à se rendre en Indo-Chine, et celle du gouvernement d'Indo-Chine, pour préserver le patrimoine indo-chinois. La \og Mission archéologique d'Indo-Chine \fg devient l'\EFEO en 1900.

\subsubsection{Organisation actuelle}
L'\EFEO se situe dans le 16\ieme arrondissement de Paris, au sein de la Maison de l'Asie, un bâtiment partagé avec l'\EPHE et l'\EHESS. L'\EFEO compte dix-huit centres de recherche en Asie, notamment à Pondichéry, Hanoi ou encore Jakarta. Les chercheurs en sciences humaines et sociales de l'\EFEO étudient les civilisations asiatiques à travers un champ pluridisciplinaire (histoire, philologie, sciences de la religion, archéologie...). L'\EFEO collabore avec le \cdf pour le projet \COREL. 

\subsection{Les infrastructures de recherche (l'\IR* Huma-Num et Data Futures)}
\subsubsection{L'\IR* Huma-Num}
L'\IR* Huma-Num est une infrastructure de recherche dans le domaine des humanités numériques et accompagne les chercheurs en sciences humaines. Elle prend appui sur les principes \fair des données et la science ouverte. Son objectif est de permettre aux chercheurs d'accéder librement aux données de la recherche, mais aussi de garantir leur pérennité et leur qualité. 

L'\IR* Huma-Num est partenaire du projet \LSC dont est issu le projet \COREL. 

\subsubsection{Data Futures}
Data Futures est une entreprise à but non-lucratif qui aide à la préservation et à l'accessibilité des données de la recherche. Elle participe à des projets d'humanités numériques à travers les sciences de la vie et les sciences humaines et sociales.

Elle est partenaire du projet \COREL et du projet précédent \EPJ. Le projet fait appel à ses prestations de manière ponctuelle pour le traitement des données.

\section{Présentation du projet \COREL}
\subsection{Description du projet}
Le projet \COREL est un projet de recherche en humanités numériques, qui vise à reconstituer l'histoire du droit de la Chine impériale tardive sous la dynastie Qing (1644 - 1911). Ce projet prend appui sur deux projets précédents, les projets \LSC et \EPJ, auxquels le responsable scientifique du projet, M. Frédéric Constant, a également participé. À partir des sources juridiques de la bibliothèque des hautes études chinoises, de leur édition en ligne par le projet \LSC et des numérisations et annotations des sources par le projet \EPJ, le projet \COREL propose à la fois une édition en ligne des textes de lois et une reconstitution de sources partielles grâce au numérique

Le projet a obtenu un financement \CollEx-Persée pour deux ans, jusqu'au mois d'octobre 2024. 

\subsection{L’objectif du projet}
Le projet \COREL a pour objectif de préserver et valoriser un corpus de sources historiques du droit chinois et de les mettre à la disposition des chercheurs, qu'ils soient historiens du droit ou sinologues. Proposer l'agrégation de ces sources partielles et qui se complètent les unes les autres pour faciliter le travail des chercheurs est au c\oe ur de ce projet, c'est pourquoi seuls des outils open-source et bien documentés, consacrés à la publication scientifique numérique et appuyés par une communauté de chercheurs en humanités numériques, sont envisagés pour réaliser les livrables.

\subsection{Les livrables du projet}
Plusieurs livrables sont attendus pour la fin du financement \CollEx-Persée en 2024. Le premier livrable est un site internet qui recrée partiellement celui du projet \LSC en proposant une édition scientifique numérique des codes légaux ainsi que des compilations des codes.
Une fonctionnalité appelée \code sera intégrée au site web afin de reconstituer la législation chinoise entre 1644 et 1911. Ce \code permet de recréer artificiellement la législation à partir l'agrégation des différentes sources du corpus. Pour une année donnée, l'utilisateur pourra consulter toutes les lois en vigueur, ordonnées selon les normes de composition d'un code légal sous la dynastie Qing (divisions en chapitres, lois secondaires classées sous la loi principale à laquelle elles sont rattachées...)

Le projet souhaite aussi intégrer des visualisations permettant de retracer la généalogie d'une loi, sa promulgation, ses modifications et éventuellement ses fusions, divisions ou encore abrogations lorsqu'elles existent. Ces visualisations seront accessibles directement depuis l'édition en ligne, en cliquant sur le titre de la loi dont l'utilisateur souhaite consulter la généalogie. 

Soumis à un financement limité dans le temps, le projet \COREL s'engage auprès de \CollEx-Persée à fournir un \POC de ces livrables, sur un jeu de données réduits si le temps ne permet pas la réalisation de tous ces livrables. Trois lois seront choisies pour représenter le corpus dans son intégralité, avec des cas de figures complexes permettant d'illustrer la faisabilité de la reconstitution de la législation chinoise impériale.


\section{Les projets de recherche antérieurs}
\subsection{Le projet Legalizing space in China}

Le projet \LSC a reçu un financement de l'\ANR de 2011 à 2015. L'objectif de ce projet est de réunir les sources du droit chinois (mais également d'autres pays d'Asie : Mongolie, Corée, Japon, Vietnam) dans une édition en ligne trilingue (chinois, français et anglais). 

Le site web du projet \LSC réunit donc un grand nombre d'éditions en ligne de codes légaux et de compilations, ainsi que leurs numérisations au format PDF. \LSC propose également des tableaux synoptiques des codes, un glossaire qui permet à la fois l'étude des termes juridiques chinois, mais aussi l'harmonisation du travail de traduction, un index des noms de lieux et de personnes et des cartes de la Chine en relation avec les textes de lois. 

Le projet \COREL s'appuie sur une partie des sources utilisées par le projet \LSC et leur édition numérique. Cette édition a été réalisée grâce à des documents encodés en \XML. Ces sources numériques sont reprises dans le cadre du projet \COREL comme socle à la création d'un nouveau jeu de données pour la publication en ligne. 

\subsection{Le projet Emerging procedural justice}
Le projet \EPJ a obtenu un financement Arqus European University Alliance en 2020. Ce projet s'est effectué en collaboration avec la société Data Futures et a permis de déposer sur un serveur \IIIF les numérisations des sources, puis de les annoter et les segmenter pour mettre en avant la structure des textes ainsi que les relations qui existent entre les lois. Ces annotations ont été réalisées à partir d'un fichier de recensement exhaustif des relations entre les lois des différents codes. 

Dans le cadre du projet \COREL, l'export de ces numérisations au format \JSON sont utilisées pour enrichir le jeu de données du projet et retracer la généalogie des lois.  
            
            
        \clearemptydoublepage
        
        \chapter{Le corpus}
        Le projet \COREL s'inscrit dans un champ disciplinaire particulier, au croisement de la sinologie et de l'histoire du droit. Développer l'accès à des ressources encore peu diffusées dans le domaine des humanités numériques constitue un enjeu majeur du projet. 
                    \section{Contexte historique}
    \subsection{La Chine impériale sous la dynastie Qing}

Le projet \COREL vise à étudier l'évolution du droit lors de la période de la Chine impériale tardive, de 1644 à 1911. Ces bornes chronologiques correspondent au règne de la dynastie Qing. En effet, les Qing, venant de Mandchourie, renversent la dynastie Ming (1368-1644) en prenant la capitale, Pékin. La dynastie Qing règne alors avec l'empereur Shunzhi (1644 - 1661) au pouvoir, lequel opère la transition entre les deux dynasties. Il fait éditer le premier code légal des Qing, le \dqlj, qui hérite directement du code des Ming, révisé pour la nouvelle dynastie. La modification du code légal de la dynastie précédente, selon Zheng Qin et Guangyuan Zhou, s'explique ainsi : 

\begin{quote}
    Indeed, the relatively smooth transition
 from the Ming code to the Qing code needs some explanation. The
 Ming code represented the highest legislative achievement in tradi-
 tional China, making the new rulers reluctant to abandon it. Further-
 more, the Manchu rulers had already familiarized themselves with the
 Ming law and political system before they conquered China. \footnote{\cite{qin_pursuing_1995}}
\end{quote}
C'est donc dans la continuité directe du code des Ming que va s'écrire le code légal des Qing.

Après Shunzhi, c'est l'empereur Kangxi (1662-1722) qui est à la tête du pays. Son règne se caractérise par une certaine ouverture aux sciences occidentales via des jésuites, venus en Chine pour transmettre leurs savoirs et tenter d'évangéliser la Chine. L'empire Mandchou s'étend progressivement jusqu'au milieu du XVIII\ieme siècle. Deux autres grands empereurs succèdent à Kangxi : Yongzheng (1723 - 1735) et Qianlong (1736 - 1795). En 1911, la dynastie Qing est remplacée par un gouvernement républicain.

\subsection{La législation chinoise et l’édition des codes légaux}

Les textes de lois chinois relèvent majoritairement du droit pénal et observent une structure définie et rigoureuse. Sous la dynastie des Qing, les codes légaux se divisent en sept chapitres majeurs (appelés \textit{bu}), structure déjà employée pour les textes de lois de la dynastie précédente. En effet, le code des Ming de 1397 utilise cette division en sept chapitres, dont les codes des Qing héritent. Le premier chapitre contient des propos généraux (\og Dénominations et règles \fg \footnote{Les titres de chapitres, entre guillemets, sont empruntés à la traduction proposée par le projet \LSC.}) et les six chapitres suivants sont organisés selon les six ministères (la division du gouvernement en ces six ministères étant observée depuis la dynastie Tang \footnote{Dynastie régnante en Chine de 618 à 907}) : \og Lois administratives \fg, \og Lois domestiques \fg, \og Lois rituelles \fg, etc. À l'instar du code des Ming, ces chapitres sont ensuite divisés en trente sections (en chinois \textit{men}), par exemple \og Institutions administratives \fg, \og Documents officiels \fg, etc. 

Ces chapitres et sections contiennent deux types de lois. Les lois dites principales, les \lu, sont des lois fixes qui constituent la colonne vertébrale du code des Qing (en anglais \textit{statutes}). Certaines lois principales sont directement héritées de la législation de la dynastie Tang, ou ont subi des changements mineurs de vocabulaire. Selon Derk Bodde et Clarence Morris, fixer des lois immuables d'une dynastie à l'autre relèverait d'une vision morale de la loi en Chine : 
\begin{quote}
    No doubt this continuity reflects the Chinese view of law as the codification of moral truths retaining eternal validity irrespective of time or place. \footnote{\cite{law_in_imperial_china}}
\end{quote}

Toutefois, ce propos est nuancé dans leur ouvrage puisque, de fait, moins de la moitié des lois principales demeurent véritablement immuables. Les autres \lu sont modifiées, parfois supprimées ou même créées sous la dynastie Qing, jusqu'en 1740, année de leur dernière version. De plus, les \lu sont accompagnées d'articles additionnels (ou lois secondaires), les \li (aussi appelées \textit{li}, ou en anglais \textit{substatutes}). Les lois secondaires sont des ajouts aux lois principales. Loin d'être figées, elles traitent du droit vivant et viennent préciser la loi principale à laquelle elles sont rattachées. Elles apparaissent souvent à la suite de jugements ou de cas particuliers. Les lois principales ont tendance à se réduire avec le temps. Elles sont au nombre de 436 en 1740. En revanche, les \li augmentent au fur et à mesure. Il en existe une centaine en 1740. À la fin du XIX\ieme siècle, on compte plus de 1 000 lois secondaires. 

 \section{Des sources juridiques}
    \subsection{Les éditions du code légal (1646, 1740)}

Les codes légaux des Qing sont révisés et publiés de manière plus ou moins régulière, tous les dix ans environ. Entre 1740 et 1871, date de la dernière édition du code, on compte 23 rééditions. Toutefois, toutes ces rééditions n'ont pas été conservées, ce qui laisse le corpus incomplet. Dans le cadre du projet \COREL, deux éditions du code légal sont utilisées. 

En 1646, la première édition du code légal des Qing, \dqlj, compte une centaine d'articles. Cette première édition du code s'inscrit dans la continuité du code de la dynastie Ming. Les historiens, notamment Tan Qian  qui a connu les deux dynasties, critiquent cette première édition du code qui n'a pratiquement rien de nouveau et ressemble très fortement au code des Ming. 

La seconde édition du code des Qing intégrée au corpus du projet est le code de 1740, \dq. Dans cette version du code, les \lu sont définitivement établies et ne sont plus modifiées. Comme son nom l'indique, ce texte met sur le même plan les lois principales, \lu et les lois secondaires, \li. En effet, sous la dynastie Ming, les \li n'étaient que des exemples aux lois principales. Elles prennent de plus en plus d'ampleur dans le droit chinois, jusqu'à avoir véritablement la même importance que les lois principales en 1740.

Néanmoins, ces éditions du codes, largement espacées, ne reflètent la loi qu'au moment où elles sont produites et ne prennent pas en compte les modifications qu'il y a pu y avoir entre deux éditions. C'est pourquoi, en plus des éditions des codes légaux, d'autres sources viennent nourrir le projet \COREL : des compilations des textes de lois, qui viennent compléter les zones d'ombres que laissent les codes légaux.

\subsection{Les compilations des textes de loi}

Les compilations des textes de lois sous la dynastie Qing ont une structure similaire à celle des codes légaux et suivent la structure rigoureuse en sept chapitres. Toutefois, ces documents présentent également des caractéristiques qui leur sont propres et viennent compléter les codes légaux. 

En 1871, le \genyuan est publié. Il présente les articles additionnels dans l'ordre chronologique, c'est-à-dire dans l'ordre de modification du code légal. Le \huidian paraît en 1899 et compile l'ensemble des lois en vigueur mais aussi les lois abrogées. Enfin, le \dc est un texte de 1905 qui compile toutes les lois en vigueur sous la dynastie des Qing, avec des explications historiques. Cette compilation a été établie par Xue Yuncheng pour aider à la révision du code légal. Ces trois textes sont des sources qui viennent compléter les textes légaux grâce à leur exhaustivité en présentant toutes les modifications et abrogations des lois et permettent d'en retracer la généalogie.

En plus de ces textes, il existe des recueils de cas qui expliquent l'origine des lois. En effet, les articles additionnels résultent de jugements ou de cas particuliers, il est donc possible de retracer l'origine d'une loi à une affaire, un décret ou un fonctionnaire. Toutefois, ces documents ne font pas partie du projet \COREL. Actuellement en cours de numérisation par la bibliothèque d'études chinoises, les informations supplémentaires que peuvent apporter ces sources constituent une perspective d'enrichissement des données du projet qui sera envisagée à termes. 


\section{Les sources numériques}
    \subsection{Les documents \XML}
Le projet \COREL dispose de sources numériques issues des projets précédents. Ces sources ont été pensées et produites pour deux projets différents et ne sont pas liées entre elles. La source de données principale du projet provient du projet \LSC, qui offre une édition en ligne des sources balisées en \XML. 

\subsubsection{Encodage et validation des données}
Ce balisage \XML relève d'un schéma personnalisé, créé spécifiquement pour le projet \LSC. Pour comprendre ce balisage, il est indispensable de consulter en regard l'édition \XML et le site web du projet, qui sont étroitement liés. En effet, le balisage reprend la structure des codes légaux, familière aux chercheurs, mais lie ce balisage à l'affichage \HTML. Ainsi, certaines portions des textes sont balisées ainsi : 
\begin{minted}{xml}
<p>笞刑五:
    <inf>笞者,擊也,又訓為恥。用小竹板。</inf>一十;
    <inf>折四板。</inf>二十;<inf>除零,折五板。</inf>三十;
    <inf>除零,折一十板。</inf>四十;
    <inf>除零,折一十五板。</inf>五十;
    <inf>折二十板。</inf>
</p>
\end{minted}

En comparant ce code \XML à l'affichage du site, il est possible d'établir clairement le lien entre l'encodage et l'affichage : la balise \texttt{<inf/>} permet d'afficher une partie des paragraphes en caractères bleus, dans une police légèrement inférieure. 
\begin{figure}[h]
    \centering
    \includegraphics[width=\textwidth]{images/image1.png}
    \caption{Capture d'écran du site web \LSC - affichage du code \XML dans l'encadré.}
\end{figure}

Cette utilisation de l'encodage relève du mode de travail des chercheurs sur le projet. En effet, à partir de textes issus de l'\OCR, le texte est saisi et balisé dans le logiciel Oxygen au fur et à mesure du projet, c'est-à-dire que l'encodage des données et le développement du site s'effectuent en parallèle. Dès lors, le site web devient un outil de validation de la saisie des données : ce lien entre l'encodage et l'affichage permet aux chercheurs débutants en humanités numériques de vérifier leur code en mettant le site à jour, les erreurs d'affichages se repérant plus facilement qu'un oubli dans le code \XML. 

Mettre en parallèle les étapes d'encodage et de création du site internet relève du choix des chercheurs du projet d'utiliser un schéma \XML entièrement customisé. En effet, ce schéma n'est pas documenté et aucune \DTD n'a été produite pour valider l'encodage et contraindre ce schéma. Le projet \LSC continue aujourd'hui d'être alimenté par les chercheurs du projet \COREL. Dès lors, le seul moyen de se repérer dans un schéma sans documentation ni règles de validation devient l'affichage du site internet.

\subsubsection{Structure des sources numériques}
Sans documentation claire ni schéma de validation, comprendre la structure des sources numériques et les choix d'encodage est une étape essentielle du projet. En effet, il est difficile de comprendre au premier coup d'oeil le schéma utilisé pour encoder les sources sans avoir travaillé pour le projet \LSC et sans connaissances solides des textes de lois chinois. De plus, au fil du temps, les sources numériques ont été encodées par de nombreuses personnes, contribuant au projet \LSC ou bien au projet \COREL. L'encodage n'étant pas contraint et les objectifs des deux projets étant différents, les sources numériques ont parfois évolué et l'encodage des texte peut présenter des variantes d'encodage d'un texte à l'autre, voire au sein d'un même document. 

L'encodage \LSC prend comme point de départ la structure des sources originales en chapitres, sections, lois principales et secondaires, mais vient ajouter des éléments supplémentaires selon les spécificités de chaque document. Par exemple, le \huidian propose des listes de lois classées par année. Cet élément est propre à ce texte de lois et a nécessité un encodage différent, traduit par un élément \texttt{<enum>} contenant un ou plusieurs \texttt{<item>} et une balise \texttt{<date>}.

De plus, le projet \LSC est un projet d'édition scientifique numérique trilingue. Tous les documents ont donc été encodés en chinois, français et anglais. Chaque balise apparaît donc trois fois, avec un attribut de langue différent. 
\begin{minted}{xml}
    <title lang="ch">Wuxing 五刑</title>
    <title lang="en">The Five Punishments</title>
    <title lang="fr">Les Cinq peines</title>
\end{minted}
Toutefois, ces choix manquent d'uniformité dans l'encodage, car certaines balises ne sont pas triples et contiennent des informations hybrides entre plusieurs langues :
\begin{minted}{xml}
    <title lang="ch">目錄 | Content</title>
\end{minted}
D'autre part, la traduction des textes n'a jamais été achevée pendant le financement du projet \LSC. Il existe ainsi de nombreuses balises auto-fermantes dans l'encodage \XML, qui laissent la place à une traduction qui n'a pas de garantie de voir le jour, d'autant plus que le projet \COREL n'inclut pas les traductions dans son périmètre. 

Le trinlinguisme initial du projet \LSC semble à première vue être un élément que l'on peut aisément écarter du projet \COREL. Toutefois, il a des conséquences directes dans l'encodage. En effet, outre les nombreuses balises auto-fermantes qui rendent le code dense, il est possible d'observer une substitution progressive entre la balise \texttt{<content>} censée séparer les différentes traductions et la balise \texttt{<p>} qui permet de séparer les différents paragraphes. Puisque la balise \texttt{<content>} n'est pas effective dans le cadre du projet \COREL et systématiquement laissée vide pour le français et l'anglais, son utilisation première s'est parfois perdue et s'est substituée à la balise \texttt{<p>} lorsque le texte chinois ne contenait qu'un seul et unique paragraphe. 

Par ailleurs, le choix d'éditer des documents trilingues s'est aussi reflété dans la création du schéma d'encodage. Si les éléments les plus courants, comme les titres, possèdent des noms de balises en anglais, ce n'est pas le cas de la plupart des éléments structurants des sources. Ainsi, les chapitres sont balisés grâce à l'élément \texttt{<bu>}, les sections \texttt{<men>}, etc. Certaines balises viennent également rappeler le français, comme la balise \texttt{<inf>} qui vient signaler des caractères de police \textit{inférieure}. 

Tous ces éléments aboutissent à la création de sources numériques extrêmement spécifiques et viennent restreindre la communauté qui peut contribuer au projet ou en bénéficier, car seules quelques personnes possèdent toutes les clefs de compréhension pour déchiffrer ce support de travail. 

\subsection{La numérisation des codes légaux}
\subsubsection{\IIIF et annotations}
En plus des données produites par le projet \LSC, le projet \COREL hérite aussi des sources produites par le projet \EPJ. Les numérisations des textes de lois ont été déposées sur un serveur \IIIF géré par Data Futures. Les chercheurs ont ensuite accès à un visualiseur Mirador qui leur permet de segmenter les images pour expliciter la structure des codes légaux en chapitres, sections, \lu et \li. Cette segmentation permet de mettre en valeur un caractère annonçant le début d'une nouvelle partie. 

\newpage
\begin{figure}[h]
    \centering
    \includegraphics[width=100mm]{images/image2.png}
    \caption{Numérisation du \dc, segmentation du début du \li 1-1}
\end{figure}

Ces caractères qui marquent le début d'un élément structurant du document ont ensuite été annotés par les chercheurs. Le prestataire a mis en place un tableau à remplir pour annoter les passages segmentés. 

\begin{figure}[h]
    \centering
    \includegraphics[width=\textwidth]{images/image3.png}
    \caption{Interface d'annotation du visualiseur Mirador}
\end{figure}
Cette pratique a permis aux chercheurs de produire des annotations structurées et uniformes d'un texte à l'autre. Dans ce cadre figurent notamment des informations sur la date de début de validité des lois et des liens de généalogie. 

\newpage
\subsubsection{Généalogie des lois}
Ce travail d'annotation et de segmentation des textes, toujours en cours en parallèle du projet \COREL, a pour objectif de retracer la généalogie des lois. Les annotations sont remplies à partir d'un référentiel. Ce référentiel prend comme source de départ le \genyuan. C'est un fichier texte disponible en ligne, hébergé par Data Futures. Il recense chaque loi une à une, généralement avec le lien vers la numérisation correspondante, et indique deux types de liens. Le premier est un lien d'association simple. Il indique les correspondances entre la loi du \genyuan à une loi d'un autre document. 
\begin{figure}[h]
    \centering
    \includegraphics[width=\textwidth]{images/image4.png}
    \caption{Modélisation d'un lien d'association entre deux lois}
    \label{Modélisation d'un lien d'association entre les lois}
\end{figure}

Ce lien d'association simple est à indiquer dans les rangs \textit{related} des annotations. Elles indiquent les lois associées à celle de l'image dans les autres textes de lois, chaque lettre correspondant à une source textuelle différente : h pour \huidian, g pour \genyuan, d pour \dc et c pour \dq. 

Le référentiel indique également un deuxième type de lien, qui n'est pas présent dans les annotations des images. C'est un lien d'association dirigée, qui indique des liens de généalogie entre les lois. Pour distinguer ces liens de généalogie des liens d'association simple, les chercheurs ont mis en place dans le référentiel un vocabulaire spécifique à ces liens. Les liens \og \textit{in} \fg indiquent qu'une loi est issue d'une autre, et les liens \og \textit{out} \fg indiquent qu'une loi donne naissance à une autre.

\begin{figure}[h]
    \centering
    \includegraphics[width=\textwidth]{images/image5.png}
    \caption{Modélisation des liens d'association dirigée entre deux lois}
\end{figure}

Lorsque les informations du référentiel sont retranscrites dans les annotations, elles sont stockées dans des fichiers \JSON que le prestataire fournit à l'équipe du projet. Toutefois, ces données ne sont que partiellement accessible car le projet dispose uniquement des fichiers correspondant aux annotations, sans le manifeste \IIIF, ne donnant pas accès aux métadonnées, lesquelles sont accessibles uniquement via une plateforme gérée par Data Futures aux utilisateurs connectés. Cet accès se fait uniquement en interface graphique, ces données ne sont donc pas exploitables. De plus, l'annotation des images étant toujours en cours de saisie, les fichiers \JSON sont encore incomplets, puisque les valeurs ne sont pas encore entièrement saisies. Dans cet exemple, la date de début de validité de la loi n'a pas encore été saisie. 

\begin{minted}{json}
    {
            "chars" : "",
            "format" : "text/plain",
            "@type" : "freizo:date"
         },
         {
            "chars" : "254",
            "@type" : "freizo:number"
         },
\end{minted}

Ces données, en plus d'être partielles et toujours en cours de saisie, contiennent parfois des informations qui sont encore à expliciter. C'est notamment le cas du champ de date des annotations. À l'intérieur de celui-ci, les chercheurs entrent la date de début de validité d'une loi. Pour retrouver la date de fin, il est nécessaire de passer par les liens de généalogie. À partir d'un lien \textit{in}, il faut alors déduire que la date de début d'une loi B met fin à la période de validité d'une loi A. 

Les données à disposition du projet \COREL sont donc entièrement issues des deux projets précédents. Pensées et produites à des fins différentes, ces sources numériques ne sont pas liées entre elles, en plus d'être produites dans des formats différents. 

            
        \clearemptydoublepage
        
        \chapter{Traitement et structuration d'un corpus documentaire}
                    \section{Le contexte de travail}
    \subsection{Une équipe issue de la collaboration inter-institutionnelle}

L'équipe du projet \COREL est composée de deux chercheurs : M. Frédéric Constant, chercheur de l'Université Côte d'Azur, responsable scientifique du projet et M. Luca Gabbiani, chercheur de l'\EFEO ; et d'un ingénieur, M. Vincent Paillusson, responsable informatique de l'\EFEO et gestionnaire à la Maison de l'Asie. La responsable administrative du projet, Mme Anne Chatellier, est la directrice des réseaux et partenariats documentaires du \cdf. Le projet s'inscrit ainsi dans un contexte de collaboration inter-institutionnelle et se trouve soumis à des contraintes à la fois géographiques, les membres de l'équipe n'étant pas réunis sur un même lieu de travail, et organisationnelles, puisque l'équipe ne travaille pas à temps plein pour le projet \COREL. 

Rattachée administrativement au \cdf lors du stage, j'ai moi-même été confrontée à ces contraintes depuis un lieu de travail différent des autres membres de l'équipe. Les échanges quotidiens du projet se déroulent majoritairement en distanciel, par mails ou visioconférences et tous les membres de l'équipe n'ont pas toujours la possibilité de se rendre disponible. Des réunions en présentiel, au \cdf ou à l'\EFEO peuvent également être organisées, mais doivent être programmées jusque plusieurs semaines en amont afin d'anticiper les obligations professionnelles de chacun. L'organisation de ces réunions aboutit généralement à des réunions longues, d'une durée supérieure à deux heures, car en plus d'être motivées par un sujet précis, elles représentent des moments rares d'échanges de l'équipe au complet et permettent de problématiser ou de résoudre certaines questions qui dépassent le cadre de la réunion. De plus, les chercheurs sont fréquemment en déplacement professionnel à l'étranger, ce qui demande un niveau d'adaptation supplémentaire : il est nécessaire d'anticiper, en plus des contraintes professionnelles, le décalage horaire. Ces contraintes d'organisation sont exigeantes pour l'équipe et leur demandent parfois de prolonger leurs journées de travail au-delà de leurs horaires habituels, complexifiant davantage les échanges hybrides.  

\subsection{Le travail déjà amorcé}
Dès mon entrée en stage, j'ai été confrontée à ces contraintes d'organisation. J'ai dû faire preuve d'une assimilation rapide des connaissances en découvrant le domaine de l'histoire du droit chinois et comprendre rapidement l'état du projet lors des réunions organisées les premiers jours, qui ont permis de réunir l'équipe du projet mais aussi des chercheurs de ce domaine d'études. Une quantité importante d'informations m'a été donnée sur une période courte de deux jours et j'ai ensuite dû apprendre à travailler à distance avec l'équipe, avec une grande place laissée à l'autonomie.

\subsubsection{Les différentes étapes de travail}
Appréhender les différentes sources, originales ou numériques, a été une difficulté supplémentaire. Le projet dispose d'un corpus assez restreint de cinq textes de lois, mais les données numériques qui en résultent sont multiples et hétérogènes. Il a donc fallu identifier en premier lieu les différentes étapes du travail déjà effectué. Les codes légaux ont d'abord été numérisés par la bibliothèques d'études chinoises puis océrisés. Les données de l'\OCR ont ensuite été encodées en \XML et publiées sur le site \LSC. L'encodage est encore en cours et le texte de l'\OCR est en train d'être corrigé. Les numérisations des codes ont ensuite été confiées au prestataire Data Futures et téléchargées sur un serveur \IIIF. Ces images ont d'abord été segmentées puis annotées. Les annotations sont en cours de corrections et d'enrichissement. 

\subsubsection{Les besoins émergents}
La présentation qui m'a été faite du projet et du travail déjà amorcé a permis aux chercheurs d'évoquer par la même occasion les besoins du projet. De ces données multiples dans des formats différents et non liés entre eux, a émergé l'axe principal du stage : comment utiliser les données des projets précédents et les adapter au projet \COREL afin de pouvoir commencer la réalisation des livrables attendus ? De prime abord, il m'a été difficile de séparer clairement les données issues des différents projets et de comprendre qu'il n'existait pas de données spécifiquement produites pour le projet \COREL. Toutefois, il était clair dès les premières réunions que les chercheurs ne savaient pas comment utiliser les productions \LSC et \EPJ pour le projet et que dans le cas du schéma \XML, les données n'étaient pas adaptées et freinaient le bon déroulement du projet. En effet, le schéma figé du projet \LSC ne répondait plus aux besoins des chercheurs qui souhaitaient par exemple ajouter des commentaires aux textes de lois déjà encodés. De plus, ce schéma se révélait trop peu flexible pour encoder les liens d'association entre les lois. 

Outre la nécessité de réadapter les données pour le projet, les chercheurs ne disposaient pas non plus d'outils open-source pour réaliser le projet. Cette problématique s'illustre en particulier avec le site web \LSC, laissé à l'abandon par son propriétaire. Les chercheurs du projet avaient le moyen d'enrichir les pages des codes légaux déjà implémentées sur le site internet via \FTP, mais ne pouvaient pas modifier le site, par exemple en y ajoutant de nouvelles pages. Le schéma \XML et le site \LSC étant liés, ces deux problèmes vont de paire et résultent de la création d'un site et d'un schéma soumis à un tiers, c'est pourquoi les chercheurs, lors des réunions de réflexion autour du projet, ont insisté sur le besoin d'utiliser des outils open source, faciles à prendre en main afin de pouvoir être le plus autonomes et indépendants possible. 


 \section{Un projet en phase d’initialisation}
    \subsection{Définition de la mission de stage}
Le stage a pour objectif de fixer un cahier des charges fonctionnel qui contextualise le projet et en donne un état des lieux exhaustif, incluant les données produites par les projets \LSC et \EPJ. Les objectifs, les besoins et les livrables attendus à la fin du projet seront détaillés le plus clairement possible, permettant à un ingénieur ou à un prestataire de saisir immédiatement la teneur du projet. Un calendrier prévisionnel jusqu’à la fin du financement \CollEx Persée, en octobre 2024 sous la forme d’un diagramme de Gantt devra être fourni. Ce cahier des charges a pour objectif de mettre à l’écrit le plus clairement possible tous les aspects attendus du projet et d’en définir clairement le périmètre. Des maquettes du site internet et des modélisations \UML des interactions des utilisateurs (à la fois front et back-office) seront également incluses dans le cahier des charges, afin de faciliter le développement du site web et de mieux appréhender les besoins des utilisateurs. 

En parallèle de ce cahier des charges, je dois commencer à mettre en place une chaîne de traitement pour la transformation des documents en \XML \TEI via une feuille de style \XSL. Les données présentes dans le document \XML devront être transformées en un document \TEI valide. Cependant, toutes les données nécessaires à la réalisation du projet ne sont pas contenues dans les documents \XML source. Un échantillon d’un document \TEI sera donc proposé afin de montrer à quoi ressembleront les documents finaux. La rédaction d’une ODD permettra de venir expliquer les pratiques d’encodages recommandées pour ce corpus. Les livrables produits pendant le stage permettront de définir les données nécessaires à la conception du site internet attendu et d’initialiser la phase de réalisation du projet. 


\subsection{Les enjeux de la mission}
L'enjeu principal de la mission est de pouvoir lancer la phase de démarrage du projet. En effet, lors de mon entrée en stage, le projet est ralenti par deux problématiques majeures : l'inadaptation des données au projet et leur absence de lien entre elles, et le besoin de rédiger un cahier des charges fonctionnels pour définir clairement les livrables attendus et délimiter le périmètre du projet, qui possède de nombreuses perspectives d'évolution. Ces deux aspects de la mission de stage sont essentiel afin de permettre aux chercheurs de travailler sur la réalisation des livrables et non plus l'enrichissement de données dans des formats inadaptés. Les étapes de nettoyage et de préparation des données sont essentielles à la bonne réalisation du projet et le stage doit permettre de mettre en place la chaîne de traitement des données et d'établir un modèle de données idéal pour l'ajout de données supplémentaires, qui n'ont pas été produites lors des projets précédents et sont propres aux objectifs du projet \COREL.

De plus, le cahier des charges fonctionnel devra inclure une partie de préconisation d'un outil open-source que les chercheurs pourront utiliser pour réaliser le projet, en tout ou en partie. Les caractéristiques de cet outil qui répondent aux besoins du projet devront être clairement explicitées pour permettre à un prestataire d'en saisir les enjeux et de mieux définir les fonctionnalités du back-office qui sont nécessaires aux chercheurs pour alimenter le site internet (ajout, modification ou suppression de contenu notamment). À l'issue du stage, les livrables produits doivent permettre à l'équipe de poursuivre la préparation des données avec les données supplémentaires apportées par le prestataire et/ou les chercheurs et de commencer à réaliser le site internet. 

\section{Les contraintes inhérentes au projet}
    \subsection{S’inscrire dans la continuité de deux projets de recherche}

Héritier de deux projets précédents, le projet \COREL utilise des productions antérieures de la recherche, ce qui nécessite d'adapter les données récupérées. Cela constitue une contrainte inhérente au projet, puisque les données ne sont pas pensées pour la réalisation des livrables. En effet, bien que les sources matérielles soient à l'origine du projet, ce sont les sources numériques qui sont utilisées pour recréer la législation. Le projet doit donc s'adapter à ces nouvelles sources, contrairement à un projet de recherche qui utiliserait comme matière première les sources originales, comme c'est le cas du projet d'édition numérique \cordel de l'Université de Genève. En effet, ce projet d'édition d'oeuvres issues de la littérature de colportage espagnole a conçu une chaîne de traitement à partir des sources originales, ce qui a permis au projet de produire des données pensées spécifiquement pour leur projet d'édition numérique. 

Cette différence de matière première aux fondations d'un projet n'est pas sans importance et présente un enjeu conséquent pour le projet \COREL. En effet, la réadaptation des données demande de repenser les sources numériques. Cette étape est d'autant plus difficile pour le projet \COREL, qui est l'héritier direct de \LSC et \EPJ : les chercheurs ayant travaillé sur les trois projets ont pensé le projet \COREL comme l'aboutissement des deux projets précédents, qui servaient la préparation des données. Dès lors, repenser les données sous une autre forme, à travers le prisme unique du projet \COREL, et non une vision d'ensemble des trois projets, requiert de rompre en partie cette continuité d'un projet à un autre. 

De plus, cette particularité du projet \COREL engendre une réflexion sur le périmètre. Bâtir sur les fondations des projets précédents, en étendant le périmètre pour y intégrer les objectifs du projet actuel, présente le risque de ne pas réussir à produire les livrables du projet. C'est notamment le cas du projet \LSC, qui est actuellement à l'abandon par le propriétaire du site web. Seuls les chercheurs continuent à alimenter ce site, mais ne peuvent qu'ajouter du texte aux codes légaux déjà édités. Les perspectives d'enrichissement du projet \LSC sont ainsi très limitées : comme la maintenance évolutive du site web qui est abandonnée ou la traduction des textes qui est inachevée. 

\subsection{Des données hétérogènes}

Utiliser les productions des projets de recherche antérieurs pose aussi la problématique de l'hétérogénéité des données. En effet, bien que ces trois projets de recherche se veulent complémentaires et continus, leurs données ne sont pas liées entre elles et ne servent pas le même but. Cette vision continue des projets résulte à la fois de l'équipe \COREL, puisque les chercheurs sont familiers des trois projets ; mais aussi de l'objectif du projet, qui regroupe à la fois la volonté d'éditer les textes de lois - l'objectif du projet \LSC - et de retracer la généalogie des lois - l'objectif du projet \EPJ. Le projet \COREL permet ainsi de faire le lien entre les objectifs des deux projets précédents. Toutefois, cela suppose de travailler à partir de sources de données hétérogènes. Lier ces deux sources de données de formats différents (les fichiers \XML et les annotations exportées au format \JSON), est indispensable pour réaliser l'objectif du projet. Ce lien entre dans la chaîne de traitement des données, qui doivent donc être réadaptées et liées entre elles. Bien que cela présente une contrainte supplémentaire, adapter non pas une, mais deux sources de données pour alimenter le projet ne limite pas la préparation des données à une seule solution unique et plusieurs perspectives sont envisageables : il est par exemple possible d'adapter les données en conservant leurs formats respectifs, puis en les liant ; de créer une nouvelle source de données unique, pensée pour le projet \COREL ; de privilégier l'un des deux formats et d'y intégrer les données manquantes. 

\subsection{Un projet de recherche en histoire du droit}

Outre les contraintes liées aux sources de données, le projet présente la contrainte de l'adaptation au public cible du projet. En effet, le projet \COREL est un projet porté par des chercheurs, pour des chercheurs. Dès lors, les humanités numériques sont un moyen de réaliser le projet, mais n'en sont pas la finalité. Il est important de garder ce fil conducteur durant toute la réalisation du projet, car les livrables qui en résultent sont différent selon l'orientation du projet. Il est pertinent de prendre appui sur l'exemple du projet \calendar, qui offre un outil permettant de convertir les dates chinoises (de la dynastie Qing à l'ère républicaine) en dates du calendrier grégorien. Ce projet d'humanités numériques a permis de produire un script python permettant de convertir les dates d'un calendrier à l'autre. Cet outil, produit pour les chercheurs en humanités numériques, n'aurait pas trouvé d'utilité auprès d'un public de chercheurs du droit chinois sans compétences informatiques. Or, l'objectif principal du projet \COREL n'est pas de faire avancer la recherche dans le numérique en créant un programme de reconstitution de la législation chinoise, mais de rendre accessibles et exploitables des sources du droit chinois. Le site internet, mais aussi son back-office, doivent donc être compréhensibles et utilisables par les chercheurs sans compétences techniques. L'utilisation d'outils open-source, en interface graphique, est un aspect essentiel de la bonne réalisation du projet, afin que les chercheurs puissent ensuite développer le site web en autonomie, notamment pour enrichir, modifier ou supprimer son contenu. 

De plus, le projet \COREL est également soumis à des contraintes inhérentes à de nombreux projets de recherche : son financement est limité dans le temps et les ressources budgétaires et humaines sont restreintes. L'équipe est constituée de deux chercheurs et d'un ingénieur qui ne sont pas disponibles à temps plein. Ainsi, tous ces facteurs nécessitent une organisation rigoureuse et une gestion efficace des moyens humains et techniques pour assurer le bon déroulement du projet. 

            
        \clearemptydoublepage


    \part{La préparation des données à l’initialisation d’un projet, enjeux de l’encodage en XML-TEI}
        \chapter{Initialisation et démarrage du projet}
                    \section{Gestion de projet}
    \subsection{Réalisation d’un état des lieux du projet}

Afin de comprendre les sources du projet \COREL et le travail effectué lors des projets précédents, il a été nécessaire de faire un état des lieux, à intégrer dans le cahier des charges.\footnote{Voir l'annexe A} En effet, les projets \LSC et \EPJ n'ont pas produit de documentation, ce qui rend la compréhension de ces projets difficile. Seuls les chercheurs ayant travaillé sur ces projets sont à même d'expliquer le travail effectué et les productions qui en découlent. De plus, sans connaissances préalables sur le droit chinois, les sources numériques produites ne sont pas faciles d'appréhension et ne se suffisent pas à elles-mêmes, d'autant plus avec un schéma d'encodage personnalisé, qui a pu se transformer avec le temps. Dès lors, l'état des lieux des projets précédents est indispensable afin de préparer les données pour le projet \COREL, d'autant plus que l'enrichissement des données \LSC et \EPJ est toujours en cours. 

En l'absence de documentation rédigée pendant le projet, ou de documents de travail détaillant les étapes de traitement des données effectuées, l'état des lieux fournit une description des sources numériques au début du projet \COREL. L'état des lieux n'est pas exhaustif sur le schéma d'encodage personnalisé du site \LSC car ces sources ne seront plus utilisées à la fin du stage mais fournit un aperçu de la transformation \XSLT et des raisons pour lesquelles le schéma \LSC ne convenait plus. L'état des lieux permet en revanche de décrire avec exhaustivité le contenu du schéma d'encodage en \TEI, résultat de cette transformation \XSLT, en plus de l'\ODD. Cette partie du cahier des charges vise à donner un premier aperçu des documents pour en comprendre la composition générale. 

Les productions du projet \EPJ sont, quant à elles, multiples et ont demandé une description plus approfondie. Appréhender ces données est un travail complexe, puisque les formats des données sont hétérogènes, même au sein du projet \EPJ. De plus, le travail étant effectué par un prestataire, l'équipe du projet n'a pas de visibilité sur les différentes étapes de travail effectuées pour aboutir à ces formats de données. L'état des lieux rédigé dans le cahier des charges reste donc une description des données fournies par le prestataire et du travail effectué par les chercheurs, mais ne peut détailler de manière exhaustive toutes les étapes de la chaîne de traitement.

L'état des lieux est organisé selon le type de données : les annotations, les métadonnées et les liens entre les lois. Ces trois types de données ne correspondent pas nécessairement à un seul format, c'est pourquoi organiser l'état des lieux un format après l'autre n'est pas pertinent et complexifie la compréhension. Les annotations, par exemple, sont accessibles au format \JSON et directement via le visualiseur Mirador, mais ont également été transformées par le prestataire au format \csv ou \tsv pour faciliter la lecture de ces informations par les chercheurs. 

Le travail de description des données s'accompagne parfois de modélisations \UML pour faciliter la compréhension.
\begin{figure}[h]
    \centering
    \includegraphics[width=\textwidth]{images/image4.png}
    \caption{Modélisation d'un lien d'association entre deux lois}
    \label{Modélisation d'un lien d'association entre les lois}
\end{figure}
En effet, le recensement de tous les liens entre les lois est un fichier assez dense et complexe et qui ne possède pas de documentation. Seuls les chercheurs du projet \EPJ sont capables d'utiliser ce recensement comme outil de travail. Or, retracer la généalogie des lois est l'un des axes majeurs du projet, ce document est donc essentiel afin de réaliser le projet. Cet exemple permet de souligner l'importance de fournir une documentation et des documents de travail clairs. En effet, deux modélisations suffisent pour expliciter ce document et permettent de comprendre facilement quels types de liens sont possibles entre les lois. L'établissement d'un cahier des charges et la production d'une documentation pendant le stage sont donc essentiels pour la suite du projet.  

\begin{figure}
    \centering
    \includegraphics[width=\textwidth]{images/image6.png}
    \caption{Capture d'écran du fichier de recensement des liens entre les lois.}
\end{figure}

\subsection{Définir les livrables attendus du projet}

L'établissement d'un cahier des charges fonctionnel\footnote{Le contenu du cahier des charges fonctionnel est préalablement décrit dans le chapitre 3.2.1.} permet également de définir clairement les livrables attendus. Le projet \COREL n'avait jusqu'alors pas de descriptif précis des livrables attendus en fin de projet. Lors de la candidature à l'appel à projets pour le financement \CollEx Persée, les chercheurs ont défini comme livrable un \POC sur trois lois représentatives du corpus pour créer un \cv. Toutefois, ce livrable est considéré par l'équipe du projet comme un engagement minimal, et le périmètre du projet est en réalité plus large. 

Une grande partie du stage a donc été consacrée à définir ce périmètre, en assurant la cohésion entre le projet de recherche et sa réalisation technique, en prenant en considération la volonté des chercheurs d'élargir leur champ de compétences en participant à certains aspects numériques du projet, comme l'encodage par exemple. Il a donc fallu dans un premier temps rassembler les idées de chaque membre de l'équipe lors de réunions, tout en gardant à l'esprit le besoin d'utiliser des outils et méthodes accessibles aux chercheurs, c'est-à-dire sans pré-requis techniques. Lors des réunions de conception du projet, il s'est avéré que les chercheurs souhaitaient un livrable bien plus étendu que le \POC : une édition en ligne la plus exhaustive possible (ajout de commentaires, entités nommées...), la reconstitution de la législation entre 1644 et 1911 et également des visualisations interactives pour retracer la généalogie d'une loi. Ces livrables n'avaient toutefois pas été clairement définis car il était important pour l'équipe de rester flexible durant toute la durée du projet. Toutefois, cette manière d'envisager le projet présente un risque conséquent. En effet, sans livrable défini ni périmètre, le projet peut ne pas être réalisé dans le temps imparti du financement et laissé à l'abandon comme c'est le cas du site web \LSC. Le cahier des charges fonctionnel a ainsi permis aux chercheurs d'exprimer toutes leurs attentes et de fixer clairement le périmètre, en dialogue avec l'ingénieur du projet. 

\subsection{Garantir la flexibilité du projet}

Ce travail de définition des besoins et des livrables en sortie du projet ne signifie pas pour autant que le projet manque de flexibilité et lors des réunions, il est clairement apparu que les besoins du projet étaient hiérarchisés. En effet, une édition en ligne simple et la reconstitution de la législation sont les deux éléments les plus importants pour assurer la bonne réalisation du projet. Toutefois, le cahier des charges ne se limite pas à ces deux aspects du projet afin d'englober à la fois les besoins secondaires comme le balisage des entités nommées ou les visualisations, mais aussi les perspectives d'enrichissement futures. Trois niveaux du livrable final ont donc ainsi été définis : en premier lieu, tous les besoins hors périmètre ont été écartés du projet. Cela concerne les données dont le projet ne dispose pas, notamment les recueils de cas qui sont actuellement en cours de numérisation par la bibliothèque d'études chinoises. Ces données sont intéressantes du point de vue scientifique et apportent une plus-value aux textes de lois, mais ne sont pas indispensables dans le cadre du projet. Elles feront l'objet d'un autre projet ou seront ajoutées après l'échéance du financement. 

Il a ensuite fallu déterminer quelles données étaient primordiales pour la réalisation de l'édition en ligne et le \cv, et lesquelles seraient intéressantes à ajouter si le temps le permet. Le site \LSC permet de démontrer que les données en l'état sont suffisantes pour publier une édition en ligne. Toutefois, les données nécessaires au \cv ne figurent que dans les annotations et ne sont pas exhaustives. La priorité est donc d'ajouter ces données manquantes dans l'encodage \TEI, notamment les dates de validité de chaque loi et des identifiants uniques permettant de les lier aux annotations. Les ajouts de commentaires ou d'entités nommées, quant à eux, sont des éléments secondaires, mais qui font partie du périmètre car ils concernent les textes de lois déjà édités, et qu'une partie de ces données a déjà été balisée dans le \XML. 

Déterminer clairement les livrables attendus en hiérarchisant les besoins du projet a ainsi permis de lancer la phase de préparation des données sans écarter la possibilité de poursuivre le projet après le financement : les perspectives d'enrichissement sont détaillées dans des documents de travail clairs, qui permettront aux chercheurs ou aux ingénieurs de venir enrichir les données avec ces informations supplémentaires ultérieurement.

 \section{Importance de l’étape de préparation des données}
    \subsection{La diversité des données des projets précédents}

Une fois les données essentielles identifiées, il est possible de commencer à préparer les données. Cette étape est primordiale afin de réaliser le projet. En effet, lors du stage, la perspective de réaliser le \POC a été évoquée, mais a été écartée des missions de stage assez rapidement, car trop de données étaient manquantes. La priorité a donc été donnée au nettoyage, à l'adaptation et à la complétion des données. 

\subsubsection{Nettoyage des données}
La première partie du travail a permis de décider, en accord avec les chercheurs, quelles données issues des projets précédents étaient à écarter pour le projet \COREL. Le nettoyage des données consiste à corriger les données, notamment les erreurs. Toutefois, dans notre cas de figure, il est important de marquer une rupture entre les données des projets antérieurs et les données du projet \COREL. Dès lors, le nettoyage des données a essentiellement consisté à supprimer les données superflues, qui n'apportent aucune valeur ajoutée aux documents. Le travail de correction des erreurs, quant à lui, a été effectué par des employés du \cdf, car il est indispensable de lire le chinois pour cette tâche. Il est donc possible de considérer deux phases distinctes du nettoyage des données : le travail de correction de l'OCR et le travail de \og correction \fg de l'encodage et de tri des informations.

L'étape de nettoyage s'est donc effectuée tout au long du stage, en supprimant petit à petit les aspects de l'encodage qui ne servaient pas les objectifs du projet, notamment le passage d'une édition trilingue à une édition chinoise uniquement.\footnote{À ce propos, voir le chapitre 2.3.1.} En effet, dans un souci d'exhaustivité, l'équipe du projet souhaitait conserver un maximum d'informations dans l'encodage, afin de pouvoir réutiliser les sources pour d'autres types de projets par exemple. Cette problématique est récurrente dans les projets d'édition scientifique numérique, car le numérique offre des possibilités plus larges que le support papier, lequel contraint forcément l'éditeur à restreindre la volumétrie de son édition : 

\begin{quote}
    L’édition numérique permet de donner un grand nombre
d’informations à propos du texte. Si, dans le cadre de l’édition
papier, l’éditrice ou l’éditeur est souvent obligé de choisir
entre une approche méthodologique et une autre, le texte
numérique permet, quant à lui, de multiplier les données et de
proposer aussi des lectures et des interprétations parallèles.
Cependant il ne faut pas tomber dans le piège d’une sorte
d’idéologie d’exhaustivité. \footnote{\cite{vitali_rosati_les_2023}}
\end{quote}

Ainsi, malgré l'exhaustivité permise par le numérique, il était important de nettoyer les données issues des projets précédents pour ne pas tomber dans ce \og piège \fg et éviter les écueils qu'a déjà connu, notamment, le projet \LSC, vaste entreprise dont l'objectif n'a pas été pleinement atteint. 

\subsubsection{Adaptation des données}
Les étapes de nettoyage et d'adaptation des données se sont déroulées en parallèle l'une de l'autre. En effet, le nettoyage des données fait partie du travail d'adaptation des données au projet \COREL, qui consiste à rendre les données non seulement exploitables pour le projet, mais surtout \textit{pensées} pour lui. La différence entre ces deux états des données est notamment démontrée par l'étape de nettoyage : dans les faits, laisser les balises de traduction anglaise et française du schéma \LSC n'est pas bloquant pour le projet. C'est d'ailleurs d'après ce postulat qu'a démarré la phase de préparation des données. Toutefois, outre la problématique d'exhaustivité, soulevée dans le point précédent, ces données supplémentaires n'apportent aucune plus-value à l'encodage, d'autant plus qu'elles sont pour la plupart vides, représentées par des balises auto-fermantes : \texttt{<content lang='fr'/>}. Ignorer ces ajouts n'est pas impossible en soit, mais pose deux soucis majeurs : le premier, purement pratique, est qu'inclure des balises vides dans l'encodage \XML ralentit le processus de préparation des données. En effet, les données \XML ont été transformées en \TEI, notamment pour que le schéma d'encodage devienne interopérable. Dès lors, ignorer ces balises superflues est impossible, puisque cela reviendrait à produire un document non valide, qui ne respecte pas le standard de la \TEI. Dans le cadre du projet, deux solutions ont été considérées : adapter, c'est-à-dire transformer ces balises selon les standards de la \TEI, ou bien nettoyer, c'est-à-dire dans ce cas de figure, les supprimer de l'encodage. La limite de temps et de financement sont des contraintes majeures de nombreux projets : dès lors, il est nécessaire d'adapter le périmètre. Ajouter des étapes supplémentaires de préparation des données n'était donc pas pertinent dans cette situation. L'exemple des balises de traduction est également représentatif du deuxième problème que soulève ce type de données superflues : la question des données propres (ou \textit{tidy data}). En effet, bien que les corrections de l'\OCR contribuent à rendre les données propres, un encodage verbeux, où toutes les balises sont triples pour envisager une édition trilingue qui ne fait partie ni du périmètre du projet, ni des perspectives d'enrichissement, rendent le code difficilement compréhensible. Plutôt que de servir le projet, ces balises le desservent, c'est pourquoi le nettoyage des données a en réalité consisté en l'extraction des données utiles au projet, en laissant de côté les informations spécifiques au projet \LSC.  


\subsection{Établissement d’un écosystème de données}

Le corpus étant encodé en \XML et en envisageant la perspective de publier les données via la plateforme \tp, le choix de données 100\% \XML pour le projet semble pertinent. Ainsi, il a été décidé d'ajouter les données extraites des fichiers \JSON dans l'encodage. Cette solution permet également d'utiliser uniquement la base de données \XML comme source de données, ce qui facilite la gestion des sources et ne demande pas de lier deux sources de données différentes. Cela est réalisable pour produire deux aspects du site web : l'édition en ligne et la reconstitution de la législation (le \cv). Cependant, une fois les livrables clairement définis, cette solution ne s'est pas révélée pleinement satisfaisante, notamment pour réaliser les visualisations. En effet, bien que cette partie du projet soit un enrichissement envisagé, situé au-dessous de l'édition et de la reconstitution de la législation dans la hiérarchie des besoins, elle entre tout de même dans le périmètre et doit être prise en considération. Afin de réaliser des visualisations interactives, la solution la plus courante est de recourir au \JS. Dès lors, conserver les fichiers \JSON et les lier aux sources \XML s'est avéré plus pertinent pour cet aspect du projet. 

Une solution hybride a donc été choisie afin de faciliter la réalisation de chaque partie du livrable. Les données nécessaires à édition en ligne ainsi qu'au \cv seront contenues dans le \XML, tandis que les données relatives à la généalogie des lois, déjà présentes dans le \JSON, permettront de réaliser les visualisations. En effet, tout comme transformer les données des anciens projets demandaient des étapes de traitement des données supplémentaires, implémenter dans les données des liens entre les lois représentait un travail fastidieux en plus d'être injustifié, puisque le recours au langage \JS privilégie le format \JSON. 

Dès lors, l'établissement d'un écosystème de données, afin de lier deux formats de données différents, a été nécessaire. Une fois les données des fichiers \JSON complétées par le prestataire, des identifiants permettront de lier chaque loi à son annotation. L'intérêt d'ajouter des identifiants est double : cela permet non seulement de lier les deux sources de données entre elles, mais également de permettre le dédoublonnage des lois lors de la réalisation du \cv, puisque certaines lois apparaissent dans plusieurs textes. Considérer l'encodage \XML comme source de données principale, et la lier aux annotations comme sources de données secondaires, est donc l'architecture de données la plus appropriée au projet \COREL. 

            
        \clearemptydoublepage
        
        \chapter{L’encodage en XML-TEI : un format standard des données}
                    \section{Pourquoi se conformer à un standard ? }
    \subsection{Un langage recommandé pour l’édition scientifique numérique}

Les corpus encodés en \TEI sont de plus en plus nombreux dans les projets de recherche en humanités numériques. La \TEI permet de préconiser des standards d'encodage au format \XML. Ces préconisations sont documentées dans les \textit{guidelines} disponibles en ligne et permettent d'assurer l'interopérabilité des données encodées en \XML pour l'édition scientifique numérique, en privilégiant un encodage sémantique afin de décrire les documents. Ce standard permet de s'adapter à un grand nombre de documents, nativement numériques ou transpositions de sources matérielles, et offre donc de nombreuses possibilités d'encodage et un niveau de personnalisation élevé, ce qui explique son utilisation de plus en plus majoritaire dans les projets d'édition scientifique numérique.

\begin{quote}
    La TEI met l’accent sur ce qui est partagé par tous les types de documents, qu’ils soient représentés physiquement sous une forme numérique sur un disque ou une carte mémoire, sous une forme imprimée comme un livre ou un journal, sous une forme écrite comme un manuscrit ou un codex, ou sous une forme inscrite dans la pierre ou sur une tablette de cire. Cette continuité facilite la migration du texte depuis des manifestations plus anciennes, comme l’imprimé ou le manuscrit, vers d’autres plus récentes comme le disque ou l’écran. \footnote{\cite{burnard_tei_2015}}
\end{quote}

En s'adaptant à tous types de documents, la \TEI représente un format de données idéal pour le projet \COREL : les textes de lois étant produits selon une architecture définie et régulière, ils se prêtent particulièrement bien à un encodage sémantique, qui permet de mettre en avant les éléments structurants des textes. De plus, la \TEI offre un vaste choix de balises, ce qui permet aux chercheurs du projets de pouvoir enrichir leurs sources, conformément aux objectifs du projet. Encoder les documents en \TEI garantit à la fois un encodage régulier et cohérent grâce à sa documentation, flexible et bien adapté aux sources grâce aux nombreuses possibilités d'encodage. Le passage à des documents encodés en \TEI, en plus de contribuer à l'interopérabilité des données de la recherche, assure aux chercheurs une indépendance dans les choix éditoriaux, contrairement au schéma figé du projet \LSC utilisé jusqu'à présent. Malgré la richesse de la \TEI, il est possible que l'utilisation de certains éléments ne correspondent pas entièrement aux spécificités d'un texte. Toutefois, ce cas de figure est également prévu par la \TEI. Il est en effet possible d'étendre le standard en modifiant le schéma d'encodage dans l'\ODD. Deux types de modifications de la \TEI sont alors possibles : des modifications dites \TEI \textit{conformant}, qui respectent les règles de la \TEI, ou bien des modifications qui outrepassent ces règles, bien que ces dernières ne soient pas conseillées. Que la \TEI soit étendue ou non, il est essentiel d'avoir un projet bien documenté, qui permette à d'autres utilisateurs de la \TEI de comprendre comment les données ont été structurées et quels choix d'encodage ont été faits. 

\subsection{Interopérabilité et documentation en ligne}

La \TEI a été créée en 1987 pour promouvoir un standard d'encodage des données en humanités numériques et ainsi permettre l'interopérabilité des ressources en ligne. En effet, le numérique apporte un foisonnement de données et de formats divers et souvent incompatibles entre eux, qui rendent l'échange des données difficile, voire impossible. Établir un standard de la diffusion des données en \textit{open access} afin de garantir leur accessibilité devient un enjeu majeur corrélé à l'apparition du web et à l'essor des formats propriétaires au profit des entreprises privées. Bien que les principes de science ouverte continuent de se développer aujourd'hui, il est primordial de maintenir la conformité à des standards afin de produire des données \fair et pérennes. 


 \section{La transformation en XML-TEI}
    \subsection{Une transformation adaptée aux besoins du projet}

§ Paragraphe 1

Idée :\\
Exemple :\\
Référence :\\
Transition :\\

§ Paragraphe 2

Idée :\\
Exemple :\\
Référence :\\
Transition :\\

§ Paragraphe 3

Idée :\\
Exemple :\\
Référence :\\
Transition :\\

\subsection{La transformation XSLT}

§ Paragraphe 1

Idée :\\
Exemple :\\
Référence :\\
Transition :\\

§ Paragraphe 2

Idée :\\
Exemple :\\
Référence :\\
Transition :\\

§ Paragraphe 3

Idée :\\
Exemple :\\
Référence :\\
Transition :\\

             
            
        \clearemptydoublepage
        
        \chapter{Définir le modèle de données du projet}
                    \section{Établir un modèle de données}
    \subsection{Réalisation d'un échantillon des données}

La préparation des données du projet \COREL s'effectue en coopération avec le prestataire. En effet, la société Data Futures est le gestionnaire de la plateforme d'annotation dans laquelle les chercheurs saisissent les données manquantes pour le projet, notamment les dates de début de période d'application des lois et les liens d'association et/ou d'association dirigée entre les lois. Ces informations sont présentes dans les annotations mais ont besoin d'être explicitées afin de pouvoir être ajoutées dans l'encodage : il est nécessaire d'ajouter des dates de fin d'application des lois notamment. L'équipe du projet a donc fait appel au prestataire afin de générer automatiquement les dates de fin de validité des lois à partir des liens de généalogie. Lorsqu'une loi en remplace une autre, la date de promulgation de la nouvelle loi marque la fin de l'application de la précédente. La chaîne de traitement des données est donc la suivante : les chercheurs enrichissent les données des annotations et les corrigent ; le prestataire génère automatiquement les dates de fin de validité des lois pour obtenir des bornes chronologiques ; les données sont ajoutées à l'encodage \TEI. Lors du stage, la première étape de cette chaîne de traitement était en cours de réalisation. La deuxième étape, prévue dans le calendrier du projet, indique que le prestataire fournit les données via des fichiers \JSON au mois de septembre. 

Afin d'anticiper la dernière étape de préparation des données, un échantillon des données a été réalisé sur un extrait du \dc, chapitre 6, section 25, \lu 254 et \li 1. Cet échantillon permet d'encoder les données manquantes afin d'établir le modèle d'encodage qui sera utilisé pour ajouter les données \JSON en \TEI. Dans un premier temps, il a été nécessaire d'identifier tous les types de données à ajouter et de déterminer les éléments \TEI les plus pertinents pour encoder ces informations. Cette étape a été réalisée une fois les documents \XML transformés en \TEI. Pour mener à bien le projet, il est pertinent d'ajouter dans l'encodage : 
\begin{itemize}
    \item Les bornes chronologiques pour chaque \lu et \li.
    \item Le lien vers les numérisations des textes.
    %à vérifier : est-ce que j'ai parlé au début du fait qu'on veut inclure les numérisations dans le site web, dans un visualiseur 3IF ?? mais que les ressources ne sont pas accessibles en ligne et que on ne peut pas accéder aux manifestes ? (et que l'équipe du projet ne le souhaite pas ?????)
    \item Un identifiant unique pour chaque \lu et \li permettant d'identifier dans chaque texte les lois associées.
    \item Des commentaires supplémentaires (de nature différente des commentaires officiels).
    \item L'enrichissement des données sur les entités nommées.
\end{itemize}

Ces éléments ont été choisis conjointement avec les chercheurs afin d'obtenir en résultat une édition en ligne complète et exploitable pour les chercheurs. À partir de cette liste, un modèle d'encodage à suivre a pu être mis en place. 

\begin{minted}{xml}
    <div type="substatute" n="1" xml:id="DQLL_254_1" 
         notBefore="1833" notAfter="1870">
        <pb  
        facs="https://duli-cunyi.freizo.org/mirador/book.cgi?catno=28941&amp;
        canvas=https://iiif.duli-cunyi.freizo.org/image/28941/canvas/p5"/>
            <p>一、反逆案内律應問擬凌遲之犯,其子孫訊明,
            實係不知謀逆情事者,無論已未成丁,均解交内務府閹割,發往新疆等處,
            給官兵為奴。如年在十歲以下者,牢固監禁,俟年届十一歲時,再行解交内務府,
            照例辦理。内務府大臣遇有解到閹割人犯,即遴派司員認眞看驗,並出具無弊切結,
            送交刑部,再行覆驗。如有情弊,即行奏參,務須查驗明確,再交兵部,發往新疆,
            給官兵為奴。至其餘律應緣坐男犯,並非逆犯子孫,年在十六歲以上者,
            發往新疆等處,給官兵為奴。如年在十五歲以下者,牢固監禁,
            俟成丁時再行發遣。緣坐婦女,發各省駐防,給官員兵丁為奴。
            其知情不首干連人犯,仍依律擬流。</p>
            <note type='metadata'>Notes on the law</note>
    </div>
\end{minted}

\subsubsection{Liens vers les images}
Les liens vers les images peuvent être ajoutés via l'attribut \texttt{@facs} sur les balises \texttt{<pb/>} qui indiquent le début d'une nouvelle page. Ces balises ont été ajoutées aux sources numériques afin de faciliter la pagination de l'édition numérique et permettra d'afficher l'image en regard du texte. Le projet \COREL souhaite aboutir à une éditon simple qui propose le texte et l'image correspondante à la page près. Un facsimile interactif n'est pas envisagé, c'est pourquoi seul l'attribut \texttt{@facs} a été choisi pour intégrer les images à l'encodage. Cette solution a également été adoptée pour le projet \cordel : 
\begin{minted}{xml}
     <div>
        <pb n="1" source="Moreno_001_1.jpg" 
        facs="fedora_ug8110021/full/full/0/default.jpg"/>
        ...
    </div>
\end{minted}

L'intégration des images pose toutefois un problème d'\textit{open access} qui n'a pas pu être résolu dans le cadre de mon stage. En effet, l'équipe du projet ne dispose pas des liens \IIIF des images, contrairement au projet \cordel. Une \URL \IIIF accepte plusieurs paramètres qui permettent d'afficher différentes zones de l'image à partir des pixels. Dans l'exemple du projet \cordel, l'image est affichée en entier grâce au paramètre \texttt{full}. Les liens \IIIF des numérisations des codes légaux, de même que les liens des ressources images seules (indiquées dans les fichiers \JSON en tant qu'identifiant), ne sont pas disponibles sur le web. Le prestataire ne fournit que le lien permettant d'accéder à l'image via leur serveur. Cela pose un problème d'accessibilité, en cours de résolution par l'équipe du projet, afin que les numérisations soient librement accessibles sur le web et que les liens puissent être inclus dans l'édition numérique. Dans l'échantillon, il a été décidé d'inclure le lien présent dans les fichiers \JSON, bien qu'il ne soit pas accessible pour le moment. Une fois l'accès autorisé par le prestataire, utiliser les liens des fichiers \JSON permettra d'automatiser l'ajout des liens dans l'encodage via un script. Cette solution est la plus satisfaisante pour le projet, étant donné que la majorité des informations à ajouter doivent l'être depuis les fichiers \JSON. \footnote{Ressource image issue de l'article \og C'est quoi le IIIF ? \fg de l'Université de Genève.}

\begin{figure}
    \centering
    \includegraphics[width=\textwidth]{images/iiif.png}
    \caption{Schéma explicatif d'une \URL \IIIF}
\end{figure}
 
\subsubsection{Les bornes chronologiques et les identifiants \XML}
L'ajout des bornes chronologiques est également permis par des attributs \TEI. Sur chaque \lu et \li, contenues dans des éléments \texttt{<div/>}, il est possible d'ajouter des attributs \texttt{@notBefore} et \texttt{@notAfter} afin d'ajouter les dates à l'encodage. Toutefois, cet ajout demande un élargissement des règles de la \TEI car ces attributs ne sont pas autorisés sur des éléments \texttt{<div/>}. L'ajout de ces attributs est indispensable afin de reconstituer la législation pour une année donnée. Placer ces attributs sur les balises \texttt{<div/>} est nécessaire afin de pouvoir afficher la loi contenue dans cette balise afin de créer le \cv, c'est pourquoi la modification de la \TEI a été choisie pour cet élément. En plus de ces bornes chronologiques, l'élément \texttt{<div/>} contient l'attribut \texttt{@xml:id} qui permet d'attribuer à chaque loi un identifiant unique. Cela permet de faire le lien avec les fichiers \JSON et d'attribuer aux lois associées, c'est-à-dire les lois qui sont les mêmes d'un texte à un autre, le même identifiant. 

Les données à inclure dans les attributs font le lien entre le format \XML et le format \JSON. Toutefois, d'autres informations d'enrichissement des données doivent être ajoutées. 

\subsubsection{Les commentaires et les entités nommées}
Les commentaires et les entités nommées sont des informations additionnelles, qui ne sont pas présentes dans les annotations. En effet, les commentaires ont été océrisés mais n'ont pas été ajoutés dans l'encodage \XML du projet \LSC, ni dans les annotations. Quelques informations sur les entités nommées ont été ajoutées dans les annotations, toutefois elles restent partielles et n'ont pas été corrigées. 

Plusieurs types de commentaires existent dans les textes de lois chinois. Les commentaires officiels, rédigés en même temps que le texte de loi, sont déjà présents dans l'encodage dans des éléments \texttt{<note type='official'/>}. Ils apparaissent dans les sources originales dans une police de caractère légèrement inférieure. Sur le site web \LSC, l'affichage met en valeur ces commentaires avec une couleur différente du reste du texte. D'autres commentaires, notamment d'auteurs ayant rédigés les compilations, seront ajoutés à l'édition en ligne. Elles sont représentées dans l'échantillon par des balises \texttt{<note/>}, toutefois leur type n'a pas encore été défini. À des fins explicatives, un attribut \texttt{@type} a été ajoutés sur ces balises pour indiquer que chaque commentaire doit posséder cet attribut obligatoire, avec un type défini.

De plus, des commentaires qui ne sont pas présents dans l'\OCR peuvent également être ajoutés, afin de créer dans l'édition en ligne un mode \og métadonnées \fg qui permettrait d'afficher des informations supplémentaires sur les lois. Ces commentaires ont donc un attribut \texttt{@type='metadata'}. 

Par ailleurs, l'équipe du projet souhaite également enrichir les informations sur les entités nommées, afin d'envisager dans l'édition en ligne un mode \og entités nommées \fg pour les mettre en avant. Dans l'encodage \LSC, les entités nommées sont balisées par des éléments \texttt{<personname/>} ou \texttt{<propername/>}. Le projet \COREL souhaite encoder un peu plus précisément les noms de personnes en leur ajoutant notamment un nom de rôle : 

\begin{minted}{xml}
    <persName role="governor" ref="#覺羅伍拉納">
        <roleName>福建巡撫</roleName>
        <name>覺羅伍拉納</name>
    </persName>
\end{minted}
En effet, la plupart des personnes mentionnées sont des fonctionnaires et ont donc un rôle qui peut être mis en valeur dans l'encodage. Afin d'éviter les erreurs d'encodage et de garantir sa régularité, il est pertinent d'ajouter dans le \texttt{<teiHeader/>} les informations relatives à ces entités nommées dans un élément \texttt{<listPerson/>} : 

\begin{minted}{xml}
     <listPerson>
        <person xml:id="覺羅伍拉納">
            <persName role="governor" ref="#覺羅伍拉納">
                <roleName>福建巡撫</roleName>
                <name>覺羅伍拉納</name>
            </persName>
            <note>Biographical information</note>
        </person>
        <person xml:id="何東山">
            <persName role="party" ref="#何東山">
                     何東山
            </persName>
            <note>Biographical information</note>
        </person>
        <person xml:id="何適">
            <persName role="party" ref="#何適">
                     何適
            </persName>
            <note>Biographical information</note>
        </person>
    </listPerson>
\end{minted}
Lister les entités nommées dans le \texttt{<teiHeader/>} permet de les identifier grâce à un attribut \textit{@xml:id} et donc d'afficher les informations relatives à chaque entité nommée si un mode \og entités nommées \fg est créé pour le site web. Le lien entre l'identifiant \XML et le nom de personne dans l'encodage s'effectue grâce à l'attribut \texttt{@ref}. Le modèle d'encodage des entités nommées a pu être déterminé grâce à l'exemple du projet \disco de l'INRIA. En effet, leur projet accorde une place importante aux entités nommées puisqu'ils offrent une édition numérique de correspondances : 
\begin{minted}{xml}
     <listPerson>
        <person xml:id="p0001">
            <persName>d'Estournelles de Constant, Paul</persName>
            <persName>
                <roleName type="nobility">Baron</roleName>
                <forename>Paul</forename>
                <nameLink>d'</nameLink>
                <surname>Estournelles de Constant</surname>
            </persName>
            <persName>Paul Henri Balluet d'Estournelles de Constant</persName>
            <nationality>French</nationality>
            <birth>
                <date when-iso="1852-11-22"/>
                <placeName>La Flèche (Sarthe)</placeName>
            </birth>
            ...
        </person>
    </listPerson>
\end{minted}
L'édition en ligne de \disco offre un encodage exhaustif des entités nommées, dans un fichier d'index \TEI dédié à cet usage. Dans le cadre du projet \COREL, un fichier d'index n'est pas envisagé pour le moment étant donné que le corpus est constitué d'un nombre restreint de documents et que les entités nommées ne sont pas le centre du projet. Toutefois, la liste des entités nommées pourra être transposée dans un index si l'encodage s'enrichit et donne lieu à une liste et à des informations exhaustives. Pour ajouter des informations additionnelles pour chaque entité nommée, l'échantillon propose une balise \texttt{<note/>} pour ajouter des informations biographiques, sans structure prédéfinie, qui pourront être affichées dans le mode \og entités nommées \fg. 

Pour le balisage des noms de lieux, une structure similaire a été mise en place dans l'échantillon. Une liste des lieux est présente dans le \texttt{<teiHeader/>} : 
\begin{minted}{xml}
     <listPlace>
        <place xml:id="寧古塔">
            <placeName ref="#寧古塔">
                寧古塔
            </placeName>
            <note>Information about the place</note>
        </place>
    </listPlace>
\end{minted}
À l'instar des noms de personnes, le nom du lieu possède un identifiant unique qui permet de le lier au balisage présent au sein du texte et un élément \texttt{<note/>} afin d'ajouter des informations supplémentaires sur le lieu. Les noms de lieux sont mentionnés dans les textes de loi en tant que lieu d'où provient l'origine de la loi ou bien en tant que lieu d'application. Selon le contexte, la balise \texttt{<placeName/>} porte donc l'attribut \texttt{@type} avec pour valeur \texttt{'application'} ou \texttt{'origin'}. 

L'établissement d'un échantillon des données permet ainsi de fournir un modèle d'encodage à suivre pour la suite du projet et donne un aperçu de l'état final du jeu de données \TEI. L'exercice a pu mettre en valeur les difficultés qui peuvent être rencontrées lors de l'encodage, notamment l'accessibilité des ressources ou le besoin d'étendre la \TEI, et ainsi de les anticiper et de commencer à les résoudre en amont afin de pouvoir enrichir le jeu de données dès le mois de septembre. Toutefois, il est également pertinent d'étudier les limites de cet aperçu idéalisé des données.

\subsection{Limites de cet aperçu idéalisé des données}
L'échantillon de données offre un modèle de données à suivre afin de réaliser le projet. Toutefois, il représente un aperçu idéalisé des données telles que l'équipe du projet les envisage à la fin du financement, afin d'offrir aux chercheurs une édition scientifique numérique la plus complète possible. Les contraintes de temps et de financement peuvent influer sur ce modèle de données et son niveau de complétude. L'échantillon permet alors de mettre en lumière les informations manquantes dans l'encodage à un instant T, notamment les informations qui seront essentiels pour mener à bien le projet (les bornes chronologiques et les identifiants \XML). Cet aperçu idéalisé des données peut être réalisé dans le périmètre du projet en automatisant l'encodage via des scripts : les données à ajouter depuis les fichiers \JSON pourront être ajoutées automatiquement. Cependant, le balisage des entités nommées et l'ajout des commentaires sont des données disponibles en texte brut sans balisage, ce qui nécessite un travail supplémentaire de la part de l'équipe du projet. L'encodage des commentaires est actuellement en cours de traitement, mais dans le modèle de données du projet \LSC. Cet enrichissement des données \LSC est hors périmètre du projet \COREL. Cependant, les chercheurs continuent d'alimenter les projets antérieurs, ce qui ajoute une étape supplémentaire dans le traitement des données. Les commentaires encodés en \XML devront être transformés en \TEI via \XSLT ou via un script. 

De plus, le travail sur les entités nommées est également en cours de traitement. Les rôles des fonctionnaires commencent à être identifiés dans les annotations, mais ces informations, repérées et ajoutées une à une par les chercheurs à la lecture des textes de loi, demande un travail de recherche long et exhaustif. Lors du stage, il a été envisagé d'utiliser un outil d'encodage automatique en \TEI, MARKUS. MARKUS est un outil qui permet de reconnaître les entités nommées (noms de personnes, de lieux et dates notamment) pour les textes chinois et coréens. C'est un outil open source, accessible en ligne qui prend en entrée du texte brut et place des balises \TEI uniquement sur les entités nommées. Un test avec l'un des documents du corpus a été réalisé afin d'envisager un traitement automatique de l'encodage. MARKUS, cependant, est une version bêta en cours de production. Bien qu'il balise les entités nommées, les documents \TEI obtenus en sortie ne sont pas valides car seules les entités nommées sont balisées, sans racine \texttt{<TEI/>} et autres balises obligatoires. Puisque les sources numériques devaient être transformées en \TEI afin de proposer une édition en ligne, l'équipe du projet a donc envisagé d'ajouter aux documents \TEI finaux les informations supplémentaires obtenues par MARKUS via un script. Cependant, à l'analyse des résultats fournis, la reconnaissance d'entités nommées contenait trop peu d'informations exactes pour justifier cette étape supplémentaire dans le traitement des données : utiliser les balises de MARKUS demande plusieurs étapes supplémentaires dans la préparation des données : une phase de relecture des résultats, puis l'ajout des entités nommées dans le corpus \TEI et enfin la vérification du résultat final obtenu. Le travail de correction a été estimé trop important, car beaucoup de bruit était généré par le balisage automatique. Par exemple, certains noms étrangers sont retranscris en chinois par des chiffres. Une confusion entre des chiffres et des noms de personnes pouvait donc être constatée dans l'encodage automatique. Un autre exemple de balisage par MARKUS montre un mésusage des attributs \TEI : 

\begin{minted}{xml}
    <roleName n="officialTitle">法司</roleName>
\end{minted}

L'attribut \texttt{@n}, utilisé pour la numérotation, est souvent utilisée par MARKUS comme un attribut \texttt{@type}. Cette solution n'a donc pas été adoptée car l'échéance du projet ne permet pas un laps de temps suffisant pour utiliser au mieux cet outil. 

L'échantillon de données et ce travail de réflexion sur la manière d'enrichir les données du projet a été le moyen de déterminer quelles données sont primordiales afin de réaliser le projet et lesquelles sont un enrichissement pertinent à proposer aux chercheurs sans pour autant être des données décisives pour la bonne réalisation des livrables.
%peut être rajouter un exemple de code sorti par Markus

 \section{Un schéma d’encodage}
    \subsection{La documentation de l’encodage}

En prenant appui sur le jeu de données \TEI de référence établi pendant le stage et sur l'échantillon des données, il a été possible de mettre en place une \ODD afin de documenter les sources du projet et de mettre en place un schéma d'encodage à suivre. 

Les projets d'édition scientifique numérique s'accompagnent en effet d'une documentation en prose dans l'\ODD. Cette documentation ne remplace pas les \textit{guidelines} de la \TEI mais viennent les compléter afin d'expliciter les spécifités d'encodage du corpus, la structure des documents et les choix faits par l'équipe. Cela est utile à la fois à l'équipe, qui peut se référer à des documents de travail clairs et aux utilisateurs qui consulteront les données une fois publiées. 

L'\ODD a été rédigée selon le modèle de données final que le projet souhaite obtenir. Certains éléments ne sont donc pas encore présent dans le jeu de données \TEI mais figurent dans l'échantillon à titre d'exemple. L'\ODD reste modifiable par l'équipe du projet si les choix d'encodage venaient à changer lors de la préparation des données tout en offrant un guide d'encodage précis à suivre agrémenté d'exemples tirés du corpus ou de l'échantillon : 

\begin{minted}{xml}
    <div>
        <head>
            <gi>particDesc</gi>
        </head>
        <p>
            The <gi>particDesc</gi> element contains the list of 
           <gi>persName</gi> present in the code. 
           Each <gi>person</gi> is attributed 
            <ref target="ODD_COREL.html#TEI.person">
                a mandatory <att>xml:id</att>
            </ref> 
            and a 
            <ref target="ODD_COREL.html#TEI.person">
                mandatory <att>role</att>.
            </ref> 
            Inside the <gi>person</gi> element, 
            the <gi>persName</gi> element must contain 
            <ref target="ODD_COREL.html#TEI.persName">
                the <att>ref</att>attribute.
            </ref> 
            Biographic notes about the person can be included if
                  necessary.
        </p>
        <egXML xmlns="http://www.tei-c.org/ns/Examples">
            <listPerson>
                <person xml:id="覺羅伍拉納">
                    <persName role="governor" ref="#覺羅伍拉納">
                        <roleName>福建巡撫</roleName>
                        <name>覺羅伍拉納</name>
                    </persName>
                    <note>Biographical information</note>
                </person>
                <person xml:id="何東山">
                    <persName role="party" ref="#何東山"> 何東山 </persName>
                    <note>Biographical information</note>
                </person>
            </listPerson>
        </egXML>
    </div>
\end{minted}

Chaque élément de l'encodage et l'utilisation qui en est faite sont décrits, avec la liste des éléments enfants lorsque l'élément peut contenir d'autres éléments \TEI. Les règles de validation spécifiques au corpus (le nombre de fois que l'élément apparaît, les attributs et/ou éléments obligatoires...) sont également définies. Un exemple permet ensuite d'illustrer la définition des éléments. 

L'\ODD rédigée a enfin été transformée via le scénario de transformation \textit{oddbyexample}. Ce schéma de transformation permet d'obtenir un fichier \HTML afin d'accéder au guide d'encodage.

\begin{figure}
    \centering
    \includegraphics[width=\textwidth]{images/odd.png}
    \caption{Capture d'écran du guide d'encodage en \HTML}
\end{figure}

La page \HTML de sortie présente un sommaire puis la documentation rédigée ainsi que les règles de validation du document \TEI. Une mise en page similaire à celle des \textit{guidelines} est obtenue. Des liens hypertextes permettent de naviguer dans le guide depuis la table des matières, vers la documentation en prose et les règles de validation, retranscrites sous forme de tableau comme dans les \textit{guidelines}.

\begin{figure}
    \centering
    \includegraphics[width=\textwidth]{images/odd2.png}
    \caption{Capture d'écran des règles de validation en \HTML}
\end{figure}

Le scénario \textit{oddbyexample} fournit également en sortie un fichier \RNG afin de le lier aux documents \TEI pour mettre en place un schéma de validation.

\newpage
\subsection{Le schéma de validation}

L'\ODD permet également de rédiger des règles de validation afin de faciliter l'encodage en \TEI et de mettre en lumière les informations manquantes ou mal encodées. En effet, la mise en place d'un schéma de validation permet, via le logiciel \textit{Oxygen}, de faire apparaître les erreurs via un code couleur : rouge pour les erreurs de première importance, et orange pour les manquements de l'encodage. 

Plusieurs règles d'encodage ont donc été définies à partir des documents du corpus ainsi que l'échantillon. Les textes de lois suivant une structure rigoureuse, les éléments \texttt{<div/>} ont fait l'objet de règles strictes afin de garantir que les erreurs de structure ou d'imbrication des éléments soient repérées : 

\begin{minted}{xml}
    <!-- Il doit y avoit 7 chapitres -->
    <constraintSpec scheme="schematron" ident="sequence" xml:id="rule05">
        <constraint>
            <s:rule context="tei:body">
                <s:assert test="count(tei:div[@type='chapter'])=7"> 
                Il doit y avoir 7 chapitres dans le document. 
                </s:assert>
            </s:rule>
        </constraint>
    </constraintSpec>
\end{minted}

Afin de pouvoir rédiger des règles sur certains types d'éléments \texttt{<div/>} uniquement, la syntaxe \textit{Schematron} a été utilisée. Les règles \textit{Schematron} permettent de donner dans l'attribut \texttt{@context} un chemin \xpath. Ainsi, la règle ci-dessus ne s'applique qu'aux éléments \texttt{<div/>} directement contenues dans la balise \texttt{<body/>}. Ces divisions qui correspondent aux chapitres des textes de loi sont obligatoirement au nombre de sept. La syntaxe \textit{Schematron} a ainsi permis d'indiquer que chaque chapitre doit contenir une ou plusieurs sections, devant elles-mêmes contenir des \lu. Les règles de validation garantissent que la structure des documents orignaux est bien respectée et régulière tout au long de l'encodage. Ces règles de validation sont essentielles afin de publier une édition scientifique numérique utilisable par les chercheurs, dans le respect de l'architecture des codes légaux chinois. 

Certains éléments, en revanche, sont spécifiques à notre projet d'édition et ne sont pas primordiaux afin de réaliser les livrables du projet, comme par exemple l'intégration des images en regard du texte. Ces éléments sont importants pour remplir les objectifs du projet puisqu'inclure la numérisation des codes légaux dans le site web fait partie du périmètre. Toutefois, il ne s'agit pas d'erreurs de même niveau qu'un problème de structuration des textes, c'est pourquoi deux niveaux d'erreurs ont été distingués : 

\begin{minted}{xml}
    <constraintSpec scheme="schematron" ident="facs" xml:id="rule08">
        <constraint>
            <s:rule context="tei:body//tei:pb">
                <s:assert test="@facs" role="WARN"> 
                    L'attribut facs est obligatoire 
                </s:assert>
            </s:rule>
        </constraint>
    </constraintSpec>
\end{minted}

Dans la règle ci-dessus, l'attribut \texttt{@role} permet de spécifier que la règle concernant l'ajout de l'attribut \texttt{@facs} pour l'intégration des images est obligatoire mais que le niveau d'erreur est inférieur à la règle précédente. La règle a donc un rôle d'\og avertissement \fg. Dans l'éditeur \textit{Oxygen}, elle apparaît en orange afin de la distinguer des erreurs de niveau supérieur.
 \begin{figure}
     \centering
     \includegraphics[width=\textwidth]{images/oxygen.png}
     \caption{Capture d'écran de l'affichage des erreurs sur Oxygen}
 \end{figure}

 \newpage
 C'est également dans l'\ODD qu'il est possible de modifier les règles de validation afin d'étendre la \TEI. Dans le cadre de la création d'un code légal généré automatiquement pour chaque année de la dynastie Qing, il est nécessaire d'encoder des périodes de validité pour chaque \lu et \li. Les attributs \texttt{@notBefore} et \texttt{@notAfter} ne sont pas autorisés sur les éléments \texttt{<div/>}. La rédaction des règles de validation a donc permis d'ajouter ces attributs sur les divisions souhaitées. Pour cela, les attributs ont été ajoutés dans la liste des attributs autorisés sur l'élément \texttt{<div/>} : 

 \begin{minted}{xml}
    <attList>
        <attDef ident="notBefore" mode="add"/>
        <attDef ident="notAfter" mode="add"/>
    </attList>
 \end{minted}

En plus de cet ajout, une règle \textit{Schematron} a été rédigée afin de spécifier que ces attributs sont obligatoires sur les \lu et les \li. 

Une fois le schéma d'encodage défini et les règles rédigées, la transformation via le scénario \textit{oddbyexample} a créé un fichier \RNG qui doit être lié aux documents \TEI comme ceci : 

\begin{minted}{xml}
<?xml-model href="ODD_COREL.rng" type="application/xml" 
schematypens="http://relaxng.org/ns/structure/1.0"?>
<?xml-model href="ODD_COREL.rng" type="application/xml" 
schematypens="http://purl.oclc.org/dsdl/schematron"?>
\end{minted}

L'éditeur \textit{Oxygen} prend ensuite en compte le schéma d'encodage défini et permet de souligner les erreurs selon les différents niveaux. Ces règles de validation sont essentielles afin de permettre à l'équipe du projet de poursuivre l'encodage et atteindre le modèle de données souhaité. La création de règles de validation permet de mettre en avant les données manquantes dans les documents et de guider l'ajout des données, ce qui assure un encodage régulier et cohérent au sein d'un texte, mais aussi d'un document à un autre. De plus, ce document peut s'avérer utile pour l'édition en \TEI d'autres textes de lois chinois et permet d'offrir à la fois une documentation claire et un outil de validation qui garantit l'obtention d'un document valide. Grâce à l'\ODD, le projet \COREL peut ainsi se détacher du schéma d'encodage \LSC initial et encoder directement les informations à ajouter et/ou de nouveaux documents en \TEI. Il est également possible de modifier l'\ODD pour ajouter de nouvelles règles si le projet d'encodage évolue dans le temps. 

             
            
        \clearemptydoublepage

    \part{La mise à disposition des données pour les chercheurs : comment agréger des sources partielles grâce au numérique ?}
    \chapter{Les bases de données document pour agréger des sources partielles}
                    \section{Pourquoi choisir une base de données document ? }
    \subsection{Les avantages d’une base de données document}
    
Le c\oe ur du projet \COREL est l'éditorialisation d'un corpus de textes juridiques. Une base de données orientée document permet de mettre en avant les documents au même plan que les informations qu'ils contiennent, ce qui est essentiel à un projet d'édition scientifique numérique. De plus, ces bases de données sont généralement structurées au format \XML ou \JSON. Le choix de ce type de base de données entrait donc en concordance avec les données initiales du projet qui se composaient essentiellement de ces deux formats. Par ailleurs, ce choix a également été motivé pour sa praticité : les données étant déjà structurées, la mise en place d'une base de données document ne demandait aucune étape supplémentaire, contrairement à la mise en place d'une base de données relationnelle. 

Toutefois, la mise en place d'une base de données relationnelle n'est pas à exclure sans réflexion. En effet, le projet \COREL vise à mettre en relation des textes et des lois entre eux. Une base de données sous forme de tables, avec des tables de relation, n'est pas sans pertinence pour un tel projet et aurait permis de faciliter la modélisation des liens entre les lois. La mise en relation des lois entre elles à l'aube du projet était représentée par un référencement complexe de liens hypertextes à partir du \genyuan. Afin de consulter les lois liées les unes aux autres à partir de ce référentiel, il est nécessaire de naviguer entre le serveur \IIIF et le référentiel et de multiplier les allers et retours. Une base de données relationnelle aurait permis aux chercheurs en humanités numériques de requêter directement en \SQL. 

La mise en place d'une base de données relationnelle est donc loin d'être inintéressante. Cependant, cette solution ne répondait pas pleinement aux besoins du projet. Bien que le requêtage d'une base de données, pour des documents partageant des liens complexes d'association et de généalogie, soit très intéressant du point de vue des humanités numériques, il n'aurait sans doute pas été exploité par la communauté ciblée par le projet \COREL, les chercheurs en histoire du droit chinois et les sinologues. L'objectif du projet est non seulement de rassembler et de lier ces sources fragmentées et partielles, mais surtout d'en faciliter l'accès et l'exploitation par les chercheurs. La mise en relation des textes et lois entre eux sans l'appui d'outils numériques est un travail long et fastidieux, comme l'a démontré le référentiel mis en place en regard des numérisations des sources. Une base de données relationnelle n'aurait ainsi pas pleinement servi l'objectif du projet : recréer artificiellement un code légal similaire aux sources originales, directement accessible et consultable afin de faciliter le travail de recherche. De plus, la restructuration des sources numériques en une base relationnelle aurait demandé des étapes de restructuration des données supplémentaires. 

Une base de données document, en revanche, offre la possibilité de mettre en relation les lois tout en conservant la structuration du document telle que la connaissent les chercheurs, rendant son exploitation plus instinctive pour le public cible. 

\subsection{Un système de gestion de base de données}

La gestion d'une base de données s'appuie sur un \SGBD. Dans TEI Publisher, un \SGBD est directement intégré à l'application, permettant d'administrer les données du site web (eXist-db) et une \IDE (eXide). Ces outils, à l'instar de TEI Publisher, sont disponibles en ligne à l'installation, documentation à l'appui, ou directement à l'essai, comme c'est le cas pour eXide. Si les chercheurs ne seront pas amenés à coder directement dans eXide en XQuery, le \SGBD est un outil indispensable à la gestion du site web du projet, puisque le site doit continuer d'être alimenté après l'échéance du financement. En plus de bénéficier d'une documentation solide, eXist-db possède une interface simple et facile à prendre en main pour les chercheurs.

En effet, jusqu'à ce jour, les chercheurs continuent d'alimenter le site internet du projet \LSC via \FTP depuis l'éditeur \XML Oxygen et ont donc déjà une expérience de gestion de base de données document. L'interface d'eXide, en plus d'être similaire à Oxygen, propose les mêmes fonctionnalités directement dans TEI Publisher, ce qui simplifie le processus d'ajout ou de modification des documents, puisque les chercheurs devaient naviguer entre Oxygen et le site web \LSC pour mettre à jour à la fois les documents (directement en \XML) et le site (via un bouton \og process \XML \fg). 

 \section{La reconstitution de la législation grâce au XQuery}
    \subsection{La chaîne de traitement envisagée}
L'\IDE, en plus d'offrir aux chercheurs une interface de modifications des données \XML pour mettre à jour les documents, permet également aux développeurs de l'application d'intervenir sur le code source afin de le personnaliser. En effet, le site web du projet pourra se développer essentiellement en interface graphique grâce à TEI Publisher en ce qui concerne la génération des pages \HTML pour l'édition des documents, mais la reconstitution de la législation à partir des sources nécessite de filtrer les données un peu plus précisément. 

Le langage XQuery est un langage de requête qui permet notamment d'interroger une base de données document. Dans le cadre du projet, le XQuery sera utilisé pour filtrer les données \XML par date, afin de reconstituer la législastion pour une date donnée par l'utilisateur. Cette requête, simple en apparence, demande d'intervenir directement sur le code plutôt que de filtrer par prédicat comme il est possible de le faire dans l'interface de TEI Publisher, car les résultats seront générés à la volée, selon la date entrée par l'utilisateur. Prévoir à l'avance le filtrage des données année par année, de 1644 à 1911, en utilisant un prédicat surchargerait le code de l'application et demanderait un travail trop conséquent. 

Les bornes chronologiques de validité de chaque loi étant encodées grâce aux attributs @notBefore et @notAfter, la requête permettra de déterminer si la date donnée en entrée est comprise entre les dates de début et de fin. Si c'est le cas, la loi sera affichée dans la reconstitution du code virtuel, avec à la suite les lois secondaires qui y sont rattachées, elles-aussi filtrées selon leur période de validité.

\subsection{Déterminer les bornes du corpus}

Avant de déterminer la manière de requêter en XQuery pour recréer la législation, il a fallu déterminer les bornes chronologiques réelles des lois présentes du corpus. En effet, la dynastie des Qing s'étend de 1644 à 1911, mais la législation n'a, de fait, pas brutalement été créée en 1644. La loi évolue sous l'influence des époques et se reconstruit sur les bases de la dynastie précédente. En l'occurrence, la législation des Qing est étroitement liée à celle des Ming et certaines lois présentes dans le corpus sont antérieures à 1644. Si le projet \COREL ne vise qu'à étudier la législation sous la dynastie Qing, il est toutefois impossible d'ignorer ses origines plus anciennes. Déterminer les bornes chronologiques d'un corpus d'études relève nécessairement de l'arbitraire. C'est pour cette raison qu'une étude plus approfondie du corpus a été nécessaire, afin d'établir dans un premier temps quelles étaient les bornes chronologiques réelles des lois du corpus. 

Une requête en XQuery a permis de lister toutes les dates présentes dans le \huidian, le seul document du corpus à posséder des dates à l'heure actuelle. 
\bigskip
\begin{minted}{xquery}
declare namespace tei="http://www.tei-c.org/ns/1.0";
for $date in distinct-values( doc("db/TEI_HDSLXB.xml")//tei:date/text())
order by $date
return $date
\end{minted}
\bigskip
Le résultat permet d'établir que la date la plus ancienne mentionnée dans le \huidian est 1616, soit presque trente ans avant le début de notre corpus. Bien que la recréation de la législation dans le cadre du projet \COREL concerne strictement les bornes chronologiques de 1644 à 1911, ce résultat amène à la réflexion suivante : comment retracer la généalogie complète d'une loi en limitant artificiellement les bornes chronologiques du corpus à la dynastie Qing ? Considère-t-on que ces dates arbitraires servent de délimiteurs stricts et que tout ce qui en dépasse le cadre doit être ignoré pour mener à bien le projet ? En suivant cette perspective, il serait alors logique, dans l'encodage, de remplacer toute date antérieure à 1644 par celle-ci. Ce choix pose toutefois un problème d'altération des sources. Le choix de bornes chronologiques propose à l'historien de présenter ses sources à travers un prisme, une vue subjective et personnelle, qui peut parfois sembler injustifiée.

Pour résoudre ce problème, il semble \textit{a priori} que conserver des dates antérieures au début du corpus ne complexifie pas la recréation de la législation sous les Qing. En effet, le site web peut spécifier à l'utilisateur que la génération d'un code légal artificiel n'est valable que pour des bornes chronologiques prédéfinies, en utilisant uniquement les textes de lois produits durant cette période. Toutefois, qu'en est-il des dates de fin ? Tout comme le droit n'est pas soudain apparu en 1644, il n'a pas pris fin en 1911. S'il est possible de trouver des références à la dynastie précédente dans les codes légaux des Qing, il est évident qu'aucune mention du futur n'est présente dans les textes. Pour assurer la cohérence de cette vision subjective de l'histoire, ne faudrait-il pas conserver la date réelle de fin de validité des lois ? Assurément, une telle entreprise ne peut être menée à l'échelle d'un projet : cela demanderait d'outrepasser le corpus prédéfini et d'y inclure des textes postérieurs à 1911, voire d'élargir démesurément le corpus jusque nos jours. De plus, le droit vivant n'ayant jamais de fin, un projet d'une telle envergure serait également infini. Cette problématique est au centre de l'édition scientifique numérique. Elena Pierazzo en fait part à propos de l'édition diplomatique : 

\begin{quote}
    So, we must have limits, and limits represent the boundaries within which the hermeneutic
process can develop. The challenge is therefore to select those limits that allow a model
which is adequate to the scholarly purpose for which it has been created.
\footnote{\cite{pierazzo_rationale_2011}}
\end{quote}

Cette question de cadre ne peut ainsi être résolue par le numérique, qui laisse ouvertes toutes les possibilités, et requiert une intervention humaine et arbitraire. Peut-on considérer que les dates indiquées dans l'encodage n'altèrent en rien la source, qu'elles ne sont que la nécessité d'un projet numérique ? Les réponses sont multiples et chacune n'en est pas moins valide. Dans le cadre du projet, l'édition des codes légaux n'est pas une édition diplomatique. Dès lors, il a été choisi d'utiliser des dates arbitraires pour l'encodage, afin d'assurer la bonne réalisation du projet. De la même manière, des identifiants \XML seront ajoutés à l'encodage \TEI pour permettre de dédoublonner les lois. Ainsi, une même loi du \huidian et du \dc porteront le même identifiant, peu importe le texte d'origine. Ce choix peut sembler peu satisfaisant d'un point de vue scientifique et altérer les sources, mais la reconstitution de codes légaux est artificielle et doit se considérer comme étant la production d'une source nouvelle, autre, qui n'altère en rien les sources originales. 

            
        \clearemptydoublepage
        
        \chapter{La mise en place d’un outil open source pour les chercheurs}
                    \section{Conception d'une plateforme pour les chercheurs}
    \subsection{Utilisation des sources du droit chinois en humanités numériques}

L'accès aux sources du droit chinois, au croisement entre les disciplines que sont l'histoire et le droit, est encore peu développé. Les collections de la bibliothèque d'études chinoises du \cdf contient les textes de lois du corpus du projet mais la bibliothèque étant en travaux, la consultation sur place n'est possible que sur rendez-vous. Sur le web, certains projets proposent un accès aux sources mais celles-ci restent majoritairement partielles. Le projet \LSC  propose une édition trilingue des codes légaux chinois de la dynastie Qing, cependant les données ne sont pas complètes à l'heure actuelle et les documents sont toujours en cours de saisie sur le site internet. Le projet de recherche japonais \textit{Terada's Homepage for Chinese Legal History Studies in Japan} \footnote{http://www.terada.law.kyoto-u.ac.jp/index_en.htm} de l'Université de Kyoto propose une édition du \dc uniquement. Or, l'étude du droit chinois et de son évolution nécessite la consultation simultanée des différents textes de lois. La conception d'un site web réunissant ces sources dans leur intégralité afin de faciliter l'accès aux chercheurs est donc au centre du projet \COREL. 

De plus, ces projets de recherche offrent uniquement un accès à leur édition en ligne, mais l'histoire du droit chinois est une discipline peu développée en humanités numériques. Les données et leur éditorialisation ne sont que peu diffusées en \textit{open access}. La production de données \textit{open source} par le projet \COREL cherche à permettre aux humanités numériques de s'approprier ce terrain de recherche et de favoriser le développement de projets de recherche sur la généalogie du droit. Cependant, des outils \textit{open source} se développent peu à peu dans les projets d'humanités numériques. La version bêta de \textit{MARKUS}, en cours de développement par l'Université de Leyde aux Pays-Bas, permet le balisage et le référencement des entités nommées pour les textes chinois. Dans le cadre du projet \COREL, nous avons également testé le script \textit{Chinese Calendar Tools} \footnote{https://gitlab.com/vandenbosch.nora/chinesecalendartools/-/tree/main}, qui permet de convertir les dates du calendrier luni-solaire chinois \footnote{Le calendrier luni-solaire, utilisé par plusieurs cultures, est un calendrier combinant le calendrier lunaire et solaire.} vers des dates du calendrier grégorien. Bien que l'échéance du financement n'ait pas permis d'exploiter ces deux outils \textit{open source} afin d'enrichir l'encodage des sources, ils permettent de démontrer que les études chinoises dans les humanités numériques en Europe se développent peu à peu. Le projet \COREL s'inscrit dans cette démarche de science ouverte et contribue avec l'éditorialisation du corpus de codes légaux de la dynastie Qing et la mise à disposition en \textit{open access} de ces données, au développement des projets d'humanités numériques pour les études chinoises. 

\subsection{Les enjeux de la reconstitution de la législation}

Offrir une édition scientifique numérique complète des textes légaux sous la dynastie Qing est donc un projet conséquent qui ambitionne de faciliter l'accès aux sources par les chercheurs, en concevant une plateforme unique réunissant les sources pour une consultation simultanée des textes. L'édition numérique permet à la fois de réunir les différents volumes d'un même code légal, mais aussi de proposer sur le même site web toutes les sources de droit chinois sous la dynastie Qing dont dispose le projet.

Toutefois, l'éditorialisation du corpus n'est qu'une partie du projet \COREL. En effet, l'équipe souhaite reconstituer la législation de 1644 à 1911, pour chaque année de la dynastie Qing. La recréation d'un texte de loi, généré automatiquement à la demande des utilisateurs, permettrait de faciliter l'accès aux sources à un niveau supérieur. À l'instar des nombreuses compilations rédigées sous la dynastie Qing, le \cv permettrait de compiler en temps réel toutes les lois valides pour une année donnée. Cet aspect du projet vise à faciliter les recherches de l'évolution du droit par le numérique : plutôt qu'un travail de recherche comparatif entre les différents textes, les chercheurs auront accès à la reconstitution de la législation grâce à l'agrégation de toutes ces sources, directement sur le web, en libre accès. 

Enfin, la reconstitution de la législation s'accompagne d'un travail sur la généalogie des lois. Les visualisations via les liens d'association dirigée entre les lois permettront également de faciliter le travail de recherche en retraçant la généalogie d'une loi en entier, sans nécessité de naviguer entre les différentes sources partielles. En effet, pour étudier la généalogie des lois, une étude de toutes les sources est nécessaire afin de trouver toutes les versions de la loi et ses modifications dans le temps, jusqu'à son abrogation. Aucun accès immédiat à la généalogie complète d'une loi n'est disponible dans les textes, puisque toutes les sources sont partielles et se complètent les unes les autres. Le projet \COREL ambitionne donc d'aider les chercheurs en facilitant ce travail de recherche sur la généalogie des lois. 

\subsection{Définition des besoins utilisateurs}

Le public cible du projet étant les chercheurs en histoire du droit chinois, il est possible d'établir des besoins utilisateurs précis pour le projet \COREL. D'une part, les chercheurs doivent pouvoir accéder à l'édition scientifique numérique des sources sur le site web, avec une structure des textes qui leur est familière, c'est-à-dire que l'édition des textes de lois doit se faire, à l'instar de la source originale, en chapitres et sections. Chaque chapitre, sections et les différentes lois doivent être numérotés afin de se repérer dans les textes. Les chercheurs ont aussi besoin de pouvoir naviguer entre les différents textes, le corpus se prêtant à de la consultation plutôt qu'à une lecture continue et linéaire. Une table des matières cliquable et qui indique la position actuelle de l'utilisateur doit donc être créée. 

En plus de l'édition en ligne, les utilisateurs doivent pouvoir accéder à la recréation de la législation, le \cv, afin de pouvoir accéder à un texte de loi composite pour une année entre 1644 et 1911. Un format \pdf doit être disponible au téléchargement pour être consulté hors connexion. Cette reconstitution de la législation doit respecter la mise en page d'un texte de lois, c'est-à-dire qu'il doit présenter les chapitres et sections usuels. Le besoin des chercheurs n'est pas de filtrer simplement les données par dates, mais de conserver la structure des codes afin de pouvoir consulter, par exemple, les lois selon les ministères auxquels elles sont rattachées, ou encore les \li selon la loi principale qu'elles viennent compléter. 

Enfin, une modélisation des liens d'association dirigée entre les lois est essentielle afin de faciliter les travaux de recherche et d'offrir une compréhension instantanée de la généalogie d'une loi. En effet, le fichier de référencement des liens entre les lois tels qu'il existe actuellement pour appuyer le projet demeure difficile d'appréhension à la première lecture et demande de naviguer via les liens hypertexte. La lecture de ce document, qui n'est pas continue, ne permet pas de saisir immédiatement la généalogie d'une loi. Apporter des visualisations aux utilisateurs est donc un besoin primordial afin de faciliter les recherches sur la généalogie des lois et leur évolution. 

Afin de répondre aux besoins des chercheurs, qu'ils soient utilisateurs ou administrateurs, il est important de concervoir une plateforme adaptée, en libre accès, afin de contribuer au développement de l'histoire du droit chinois en humanités numériques.

 \section{La publication de données en ligne : un travail accessible à un plus large public}
    \subsection{L’outil TEI Publisher}

%présentation et méthodologie de travail (les interviews)
§ Paragraphe 1

Idée :\\
Exemple :\\
Référence :\\
Transition :\\

§ Paragraphe 2

Idée :\\
Exemple :\\
Référence :\\
Transition :\\

§ Paragraphe 3

Idée :\\
Exemple :\\
Référence :\\
Transition :\\

\subsection{L’édition scientifique en interface graphique}

§ Paragraphe 1

Idée :\\
Exemple :\\
Référence :\\
Transition :\\

§ Paragraphe 2

Idée :\\
Exemple :\\
Référence :\\
Transition :\\

§ Paragraphe 3

Idée :\\
Exemple :\\
Référence :\\
Transition :\\

%si pas fait avant : parler des limites de tei publisher.


             
            
        \clearemptydoublepage
        
        \chapter{Assurer un outil pérenne}
                    \section{Le choix de la science ouverte}
    \subsection{Publication des données sous licence libre : la licence Etalab}

Le projet \COREL souhaite produire une édition scientifique numérique \textit{open source} et contribuer à l'enrichissement des données de la recherche. Lors de la rédaction du cahier des charges et de l'établissement d'un jeu de données en \TEI, la question de la licence à utiliser s'est posée. De nombreux projets de recherche utilisent les licences \textit{Creative Commons}, dont le site internet permet de choisir facilement une licence qui correspond aux besoins de chacun. En effet, un questionnaire à remplir permet ensuite de rediriger l'utilisateur vers la licence qui correspond le mieux à ses réponses. Les licences \textit{Creative Commons} sont également bien documentées afin de permettre à chacun de choisir au mieux la licence qu'il souhaite utiliser. 

Dans un premier temps, la possibilité d'utiliser une licence CC-BY-SA\footnote{https://creativecommons.org/licenses/by-sa/4.0/} a été envisagée par le projet. Cette licence permet d'autoriser la réutilisation des données en attribuant au projet \COREL la paternité des données originales et d'indiquer les modifications effectuées sur les données (BY), et de partager les données dans les mêmes conditions que le projet \COREL (SA, \textit{Share Alike}), c'est-à-dire en conservant la licence CC-BY-SA. 

Toutefois, le projet \COREL est un projet de recherche public. C'est pourquoi la licence Etalab\footnote{https://www.etalab.gouv.fr/licence-ouverte-open-licence/} a finalement été choisie pour le projet. Mise en place par le gouvernement français dans le cadre de l'\textit{open data}, cette licence est : 

\begin{quote}
    la licence de référence pour les administrations pour la publication de données publiques.\footnote{Ibid.}
\end{quote}

Cette licence est équivalente à la licence CC-BY et est donc compatible avec celle-ci, si le projet \COREL atteint des chercheurs en droit chinois ailleurs que dans le cadre de la réglementation française. Le choix de cette licence s'est imposé afin de respecter la licence mise en place par le gouvernement pour les institutions publiques. La réflexion autour de cette licence a également permis au projet d'envisager des conditions de réutilisation plus libres. En effet, la licence CC-BY-SA présente plus de contraintes que la licence CC-BY ou Etalab, étant donné qu'elle impose aux utilisateurs de repartager les données sous la même licence. La licence CC-BY n'impose aucune restriction et permet de partager ses données en \textit{open access}, sans autre condition que l'attribution de la paternité de l'oeuvre à un tiers. En souhaitant contribuer à l'ouverture des données et au partage des données publiques, le projet \COREL a donc choisi d'utiliser la licence Etalab en France afin de s'inscrire dans une démarche de science ouverte. 

\subsection{Utilisation d’outils open-source, maintenus par une communauté scientifique}

En plus du choix de la licence, il était essentiel pour le projet d'utiliser des outils et langages \textit{open source} et bien documentés, afin d'éviter les écueils des projets précédents. En effet, les données des projets précédents, bien que pensés pour être disponibles en \textit{open access} n'ont en réalité pas contribués à la science ouverte et à l'enrichissement des données de la recherche. Les données du projet \LSC ont été balisées en \XML afin d'utiliser un langage standard de partage des données. Cependant, le manque de documentation et de diffusion de ces données a contribué à l'établissement de données difficiles d'accès et non-réutilisables. Afin de publier des données respectant les principes \fair et rétablir l'accès aux sources du droit chinois, le projet \COREL a donc transformé ces données en \TEI, ce qui a permis d'établir un schéma d'encodage bien documenté, qui pourra être consulté par des chercheurs en droit chinois et en humanités numériques. Afin de contribuer à l'ouverture des données, il est également nécessaire de publier ces données afin qu'elles soient accessibles librement sur le web, par exemple sur une page GitHub dédiée au projet.

De plus, la plateforme \tp, maintenue par une communauté scientifique en publication de données \TEI, permet au projet \COREL d'utiliser un outil \textit{open source} et bien documenté. Cela permet d'éviter la création d'une plateforme éphémère comme le site web \LSC, laissé à l'abandon par son propriétaire, sans moyen de l'entretenir. L'usage d'un outil \textit{open source} a pour avantage d'être maintenu par une communauté entière, à l'inverse d'un site propriétaire. De plus, \tp étant fondé sur le \TEI Processing Model\footnote{Le \TEI Processing Model est le standard surlequel s'appuie l'ODD de \tp. Il est documenté dans les guidelines de la \TEI et permet d'attribuer un comportement à une balise en écrivant du code dans l'ODD.}, si la plateforme venait à ne plus fonctionner, l'\ODD générée par celle-ci serait toujours utilisable puisqu'elle s'inscrit dans les standards de la \TEI. Néanmoins, l'utilisation d'un outil \textit{open source} n'est pas le seul garant de la maintenabilité du site web du projet dans le temps, même après le financement. Il est essentiel de penser également à cette question de pérennisation des données publiées en ligne, afin de produire une véritable contribution à la science ouverte, et non une plateforme à durée de vie limitée. 

 \section{Perspectives et évolutions du projet }
    \subsection{Maintenance et hébergement}
Afin d'assurer un outil pérenne pour les chercheurs, il est nécessaire de penser en amont à l'hébergement et à la maintenance du site web. Ces deux aspects du projets ont été intégrés au cahier des charges, afin de souligner l'importance d'héberger et de maintenir le site web du projet dans le temps pour ne pas créer une plateforme qui deviendrait obsolète dans quelques années. 

Plusieurs solutions sont envisagées pour l'hébergement. Étant donné que le projet résulte d'une collaboration entre les institutions, il est possible que le \cdf ou l'\EFEO soient, l'un ou l'autre, l'hébergeur du site web. Le projet envisage également de faire appel à Huma-Num pour héberger le site. Huma-Num propose d'héberger gratuitement, cependant la maintenance reste aux frais du projet et doit être garantie afin d'obtenir l'hébergement d'Huma-Num. De plus, d'autres conditions sont spécifiées sur le site web de l'infrastructure\footnote{https://documentation.huma-num.fr/hebergement-web/}, afin d'assurer aux utilisateurs que les données soient ouvertes et interopérables. Les données et métadonnées du site doivent, notamment, être référencées dans Isidore\footnote{Isidore est un moteur de recherche mis en place par Huma-Num pour les sciences humaines et sociales, qui référence des publications scientifiques, colloques et toutes sortes de documents et permet de faire une recherche plein texte dans ces documents.} via le protocole \oai \footnote{Ce protocole permet de garantir l'interopérabilité des données grâce à un standard permettant de diffuser des données et d'en collecter.}. Bien que seul l'hébergement Huma-Num requiert obligatoirement le respect de ce standard, l'interopérabilité et l'ouverture des données sont essentiels pour l'équipe du projet, afin d'enrichir les données de la recherche et de produire un outil accessible aux chercheurs, contrairement au site web précédent qui n'est plus maintenu et ne respecte pas les principes \fair des données. Le référencement dans Isidore fait donc partie des étapes de mise en place du site web, peu importe l'hébergeur choisi. 

Par ailleurs, la maintenance envisagée pour le projet \COREL est essentiellement corrective, afin de garantir un site web pérenne dans le temps. En effet, les chercheurs souhaitent pouvoir mettre à jour le site internet avec de nouveaux documents, sur le même modèle d'encodage défini pour les textes légaux. Si les nouveaux documents \TEI respectent le schéma de validation de l'\ODD, les documents devraient s'afficher correctement grâce à l'\ODD de \tp. Ainsi, le site web du projet ne demande qu'une maintenance corrective, sans ajout de nouvelles fonctionnalités. Cette maintenance sera assurée par Vincent Paillusson et permettra de veiller à la bonne intégration des nouveaux documents sur le site. 

\subsection{Les évolutions envisagées}

L'évolution principale du projet \COREL consiste en l'ajout de nouveaux documents. En effet, des recueils de cas et de jugements sont actuellement en train d'être numérisés par la bibliothèque d'études chinoises. À termes, il est donc envisagé de les intégrer au site internet du projet et de les lier aux lois auxquelles elles sont rattachées. En effet, certains cas donnent naissance à de nouvelles lois ou articles additionnels afin d'adapter la loi à un cas spécifique. Afin d'offrir aux chercheurs un outil permettant d'étudier l'évolution et la généalogie des lois, intégrer ces recueils de cas au site web n'est donc pas sans intérêt pour la recherche. L'ajout de ces nouveaux documents nécessite de les encoder en \TEI sur le même modèle que les textes de lois. Toutefois, le modèle d'encodage ayant été créés pour les codes légaux sans prendre en compte les recueils de cas, le schéma d'encodage devra vraisemblablement être adapté aux nouveaux documents et nécessitera de modifier l'\ODD de \tp ou d'en créer une nouvelle afin de personnaliser l'affichage pour ces documents. Ces évolutions relèvent donc de la maintenance évolutive étant donné que le paramétrage de l'application réalisée avec \tp devra être modifié afin de s'adapter à ces nouveaux ajouts. 

De plus, l'ajout d'autres visualisations ont été évoquées pendant le stage, mais n'ont pas été intégrées au cahier des charges car elles outrepassent le périmètre du projet. Il est toutefois intéressant de conserver ces perspectives d'évolution du projet. Ainsi, les visualisations suivantes peuvent être envisagées comme une évolution du site web du projet : 

\begin{itemize}
    \item Établir une cartographie du code : l'organisation d'un code légal a pu changer dans le temps et certaines lois additionnelles peuvent changer de \lu de rattachement. 
    \item Réaliser des études statistiques sur les textes : le pourcentage de caractères ayant changé d'une version à une autre d'un code légal ou suivre des évolutions de vocabulaire par exemple.
\end{itemize}

Le projet \COREL envisage donc une maintenance corrective afin de maintenir le site web dans le temps et permettre aux chercheurs d'accéder aux données et à l'édition des textes librement. Toutefois, des évolutions plus importantes sont également pensées et nécessiteraient une maintenance évolutive, et probablement un financement supplémentaire afin de les mener à bien. Bien que ces perspectives d'évolution soient hors périmètre dans le cadre du projet, il est important de les envisager en amont afin de fournir aux utilisateurs un site web pérenne et des données utilisables, qui respectent les principes de l'\textit{open data} et soient consultables et réutilisables par les chercheurs et ainsi produire une plateforme utile à la recherche. 


             
            
        \clearemptydoublepage
    
    \chapterNo{Conclusion}
    \addcontentsline{toc}{chapter}{Conclusion}

\appendix
    \part*{Annexes}	
    \addcontentsline{toc}{part}{Annexes}
    \include{annexes/}

\clearemptydoublepage

\backmatter
    \printacronyms[title=Liste des acronymes,toctitle=Acronymes]
    \addcontentsline{toc}{chapter}{\listfigurename}
    \listoffigures
    \printglossary 
    \printbibliography[keyword={histoire}, title={Histoire du droit chinois}]
    \printbibliography[keyword={edition}, title={Édition scientifique numérique}]
    \tableofcontents
	
\end{document}
