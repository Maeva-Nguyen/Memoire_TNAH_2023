\section{Pourquoi se conformer à un standard ? }
    \subsection{Un langage recommandé pour l’édition scientifique numérique}

Les corpus encodés en \TEI sont de plus en plus nombreux dans les projets de recherche en humanités numériques. La \TEI permet de préconiser des standards d'encodage au format \XML. Ces préconisations sont documentées dans les \textit{guidelines} disponibles en ligne et permettent d'assurer l'interopérabilité des données encodées en \XML pour l'édition scientifique numérique, en privilégiant un encodage sémantique afin de décrire les documents. Ce standard permet de s'adapter à un grand nombre de documents, nativement numériques ou transpositions de sources matérielles, et offre donc de nombreuses possibilités d'encodage et un niveau de personnalisation élevé, ce qui explique son utilisation de plus en plus majoritaire dans les projets d'édition scientifique numérique.

\begin{quote}
    La TEI met l’accent sur ce qui est partagé par tous les types de documents, qu’ils soient représentés physiquement sous une forme numérique sur un disque ou une carte mémoire, sous une forme imprimée comme un livre ou un journal, sous une forme écrite comme un manuscrit ou un codex, ou sous une forme inscrite dans la pierre ou sur une tablette de cire. Cette continuité facilite la migration du texte depuis des manifestations plus anciennes, comme l’imprimé ou le manuscrit, vers d’autres plus récentes comme le disque ou l’écran. \footnote{\cite{burnard_tei_2015}}
\end{quote}

En s'adaptant à tous types de documents, la \TEI représente un format de données idéal pour le projet \COREL : les textes de lois étant produits selon une architecture définie et régulière, ils se prêtent particulièrement bien à un encodage sémantique, qui permet de mettre en avant les éléments structurants des textes. De plus, la \TEI offre un vaste choix de balises, ce qui permet aux chercheurs du projets de pouvoir enrichir leurs sources, conformément aux objectifs du projet. Encoder les documents en \TEI garantit à la fois un encodage régulier et cohérent grâce à sa documentation, flexible et bien adapté aux sources grâce aux nombreuses possibilités d'encodage. Le passage à des documents encodés en \TEI, en plus de contribuer à l'interopérabilité des données de la recherche, assure aux chercheurs une indépendance dans les choix éditoriaux, contrairement au schéma figé du projet \LSC utilisé jusqu'à présent. Malgré la richesse de la \TEI, il est possible que l'utilisation de certains éléments ne correspondent pas entièrement aux spécificités d'un texte. Toutefois, ce cas de figure est également prévu par la \TEI. Il est en effet possible d'étendre le standard en modifiant le schéma d'encodage dans l'\ODD. Deux types de modifications de la \TEI sont alors possibles : des modifications dites \TEI \textit{conformant}, qui respectent les règles de la \TEI, ou bien des modifications qui outrepassent ces règles, bien que ces dernières ne soient pas conseillées. Que la \TEI soit étendue ou non, il est essentiel d'avoir un projet bien documenté, qui permette à d'autres utilisateurs de la \TEI de comprendre comment les données ont été structurées et quels choix d'encodage ont été faits. 

\subsection{Interopérabilité et documentation en ligne}

La \TEI a été créée en 1987 pour promouvoir un standard d'encodage des données en humanités numériques et ainsi permettre l'interopérabilité des ressources en ligne. En effet, le numérique apporte un foisonnement de données et de formats divers et souvent incompatibles entre eux, qui rendent l'échange des données difficile, voire impossible. Établir un standard de la diffusion des données en \textit{open access} afin de garantir leur accessibilité devient un enjeu majeur corrélé à l'apparition du web et à l'essor des formats propriétaires au profit des entreprises privées. Bien que les principes de science ouverte continuent de se développer aujourd'hui, il est primordial de maintenir la conformité à des standards afin de produire des données \fair et pérennes. 


 \section{La transformation en XML-TEI}
    \subsection{Une transformation adaptée aux besoins du projet}

§ Paragraphe 1

Idée :\\
Exemple :\\
Référence :\\
Transition :\\

§ Paragraphe 2

Idée :\\
Exemple :\\
Référence :\\
Transition :\\

§ Paragraphe 3

Idée :\\
Exemple :\\
Référence :\\
Transition :\\

\subsection{La transformation XSLT}

§ Paragraphe 1

Idée :\\
Exemple :\\
Référence :\\
Transition :\\

§ Paragraphe 2

Idée :\\
Exemple :\\
Référence :\\
Transition :\\

§ Paragraphe 3

Idée :\\
Exemple :\\
Référence :\\
Transition :\\
