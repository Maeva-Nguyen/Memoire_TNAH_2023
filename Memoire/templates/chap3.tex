\section{Le contexte de travail}
    \subsection{Une équipe issue de la collaboration inter-institutionnelle}

L'équipe du projet \COREL est composée de deux chercheurs : M. Frédéric Constant, chercheur de l'Université Côte d'Azur, responsable scientifique du projet et M. Luca Gabbiani, chercheur de l'\EFEO ; et d'un ingénieur, M. Vincent Paillusson, responsable informatique de l'\EFEO et gestionnaire à la Maison de l'Asie. La responsable administrative du projet, Mme Anne Chatellier, est la directrice des réseaux et partenariats documentaires du \cdf. Le projet s'inscrit ainsi dans un contexte de collaboration inter-institutionnelle et se trouve soumis à des contraintes à la fois géographiques, les membres de l'équipe n'étant pas réunis sur un même lieu de travail, et organisationnelles, puisque l'équipe ne travaille pas à temps plein pour le projet \COREL. 

Rattachée administrativement au \cdf lors du stage, j'ai moi-même été confrontée à ces contraintes depuis un lieu de travail différent des autres membres de l'équipe. Les échanges quotidiens du projet se déroulent majoritairement en distanciel, par mails ou visioconférences et tous les membres de l'équipe n'ont pas toujours la possibilité de se rendre disponible. Des réunions en présentiel, au \cdf ou à l'\EFEO peuvent également être organisées, mais doivent être programmées jusque plusieurs semaines en amont afin d'anticiper les obligations professionnelles de chacun. L'organisation de ces réunions aboutit généralement à des réunions longues, d'une durée supérieure à deux heures, car en plus d'être motivées par un sujet précis, elles représentent des moments rares d'échanges de l'équipe au complet et permettent de problématiser ou de résoudre certaines questions qui dépassent le cadre de la réunion. De plus, les chercheurs sont fréquemment en déplacement professionnel à l'étranger, ce qui demande un niveau d'adaptation supplémentaire : il est nécessaire d'anticiper, en plus des contraintes professionnelles, le décalage horaire. Ces contraintes d'organisation sont exigeantes pour l'équipe et leur demandent parfois de prolonger leurs journées de travail au-delà de leurs horaires habituels, complexifiant davantage les échanges hybrides.  

\subsection{Le travail déjà amorcé}
Dès mon entrée en stage, j'ai été confrontée à ces contraintes d'organisation. J'ai dû faire preuve d'une assimilation rapide des connaissances en découvrant le domaine de l'histoire du droit chinois et comprendre rapidement l'état du projet lors des réunions organisées les premiers jours, qui ont permis de réunir l'équipe du projet mais aussi des chercheurs de ce domaine d'études. Une quantité importante d'informations m'a été donnée sur une période courte de deux jours et j'ai ensuite dû apprendre à travailler à distance avec l'équipe, avec une grande place laissée à l'autonomie.

\subsubsection{Les différentes étapes de travail}
Appréhender les différentes sources, originales ou numériques, a été une difficulté supplémentaire. Le projet dispose d'un corpus assez restreint de cinq textes de lois, mais les données numériques qui en résultent sont multiples et hétérogènes. Il a donc fallu identifier en premier lieu les différentes étapes du travail déjà effectué. Les codes légaux ont d'abord été numérisés par la bibliothèques d'études chinoises puis océrisés. Les données de l'\OCR ont ensuite été encodées en \XML et publiées sur le site \LSC. L'encodage est encore en cours et le texte de l'\OCR est en train d'être corrigé. Les numérisations des codes ont ensuite été confiées au prestataire Data Futures et téléchargées sur un serveur \IIIF. Ces images ont d'abord été segmentées puis annotées. Les annotations sont en cours de corrections et d'enrichissement. 

\subsubsection{Les besoins émergents}
La présentation qui m'a été faite du projet et du travail déjà amorcé a permis aux chercheurs d'évoquer par la même occasion les besoins du projet. De ces données multiples dans des formats différents et non liés entre eux, a émergé l'axe principal du stage : comment utiliser les données des projets précédents et les adapter au projet \COREL afin de pouvoir commencer la réalisation des livrables attendus ? De prime abord, il m'a été difficile de séparer clairement les données issues des différents projets et de comprendre qu'il n'existait pas de données spécifiquement produites pour le projet \COREL. Toutefois, il était clair dès les premières réunions que les chercheurs ne savaient pas comment utiliser les productions \LSC et \EPJ pour le projet et que dans le cas du schéma \XML, les données n'étaient pas adaptées et freinaient le bon déroulement du projet. En effet, le schéma figé du projet \LSC ne répondait plus aux besoins des chercheurs qui souhaitaient par exemple ajouter des commentaires aux textes de lois déjà encodés. De plus, ce schéma se révélait trop peu flexible pour encoder les liens d'association entre les lois. 

Outre la nécessité de réadapter les données pour le projet, les chercheurs ne disposaient pas non plus d'outils open-source pour réaliser le projet. Cette problématique s'illustre en particulier avec le site web \LSC, laissé à l'abandon par son propriétaire. Les chercheurs du projet avaient le moyen d'enrichir les pages des codes légaux déjà implémentées sur le site internet via \FTP, mais ne pouvaient pas modifier le site, par exemple en y ajoutant de nouvelles pages. Le schéma \XML et le site \LSC étant liés, ces deux problèmes vont de paire et résultent de la création d'un site et d'un schéma soumis à un tiers, c'est pourquoi les chercheurs, lors des réunions de réflexion autour du projet, ont insisté sur le besoin d'utiliser des outils open source, faciles à prendre en main afin de pouvoir être le plus autonomes et indépendants possible. 


 \section{Un projet en phase d’initialisation}
    \subsection{Définition de la mission de stage}
Le stage a pour objectif de fixer un cahier des charges fonctionnel qui contextualise le projet et en donne un état des lieux exhaustif, incluant les données produites par les projets \LSC et \EPJ. Les objectifs, les besoins et les livrables attendus à la fin du projet seront détaillés le plus clairement possible, permettant à un ingénieur ou à un prestataire de saisir immédiatement la teneur du projet. Un calendrier prévisionnel jusqu’à la fin du financement \CollEx Persée, en octobre 2024 sous la forme d’un diagramme de Gantt devra être fourni. Ce cahier des charges a pour objectif de mettre à l’écrit le plus clairement possible tous les aspects attendus du projet et d’en définir clairement le périmètre. Des maquettes du site internet et des modélisations \UML des interactions des utilisateurs (à la fois front et back-office) seront également incluses dans le cahier des charges, afin de faciliter le développement du site web et de mieux appréhender les besoins des utilisateurs. 

En parallèle de ce cahier des charges, je dois commencer à mettre en place une chaîne de traitement pour la transformation des documents en \XML \TEI via une feuille de style \XSL. Les données présentes dans le document \XML devront être transformées en un document \TEI valide. Cependant, toutes les données nécessaires à la réalisation du projet ne sont pas contenues dans les documents \XML source. Un échantillon d’un document \TEI sera donc proposé afin de montrer à quoi ressembleront les documents finaux. La rédaction d’une ODD permettra de venir expliquer les pratiques d’encodages recommandées pour ce corpus. Les livrables produits pendant le stage permettront de définir les données nécessaires à la conception du site internet attendu et d’initialiser la phase de réalisation du projet. 


\subsection{Les enjeux de la mission}
L'enjeu principal de la mission est de pouvoir lancer la phase de démarrage du projet. En effet, lors de mon entrée en stage, le projet est ralenti par deux problématiques majeures : l'inadaptation des données au projet et leur absence de lien entre elles, et le besoin de rédiger un cahier des charges fonctionnels pour définir clairement les livrables attendus et délimiter le périmètre du projet, qui possède de nombreuses perspectives d'évolution. Ces deux aspects de la mission de stage sont essentiel afin de permettre aux chercheurs de travailler sur la réalisation des livrables et non plus l'enrichissement de données dans des formats inadaptés. Les étapes de nettoyage et de préparation des données sont essentielles à la bonne réalisation du projet et le stage doit permettre de mettre en place la chaîne de traitement des données et d'établir un modèle de données idéal pour l'ajout de données supplémentaires, qui n'ont pas été produites lors des projets précédents et sont propres aux objectifs du projet \COREL.

De plus, le cahier des charges fonctionnel devra inclure une partie de préconisation d'un outil open-source que les chercheurs pourront utiliser pour réaliser le projet, en tout ou en partie. Les caractéristiques de cet outil qui répondent aux besoins du projet devront être clairement explicitées pour permettre à un prestataire d'en saisir les enjeux et de mieux définir les fonctionnalités du back-office qui sont nécessaires aux chercheurs pour alimenter le site internet (ajout, modification ou suppression de contenu notamment). À l'issue du stage, les livrables produits doivent permettre à l'équipe de poursuivre la préparation des données avec les données supplémentaires apportées par le prestataire et/ou les chercheurs et de commencer à réaliser le site internet. 

\section{Les contraintes inhérentes au projet}
    \subsection{S’inscrire dans la continuité de deux projets de recherche}

Héritier de deux projets précédents, le projet \COREL utilise des productions antérieures de la recherche, ce qui nécessite d'adapter les données récupérées. Cela constitue une contrainte inhérente au projet, puisque les données ne sont pas pensées pour la réalisation des livrables. En effet, bien que les sources matérielles soient à l'origine du projet, ce sont les sources numériques qui sont utilisées pour recréer la législation. Le projet doit donc s'adapter à ces nouvelles sources, contrairement à un projet de recherche qui utiliserait comme matière première les sources originales, comme c'est le cas du projet d'édition numérique \cordel de l'Université de Genève. En effet, ce projet d'édition d'oeuvres issues de la littérature de colportage espagnole a conçu une chaîne de traitement à partir des sources originales, ce qui a permis au projet de produire des données pensées spécifiquement pour leur projet d'édition numérique. 

Cette différence de matière première aux fondations d'un projet n'est pas sans importance et présente un enjeu conséquent pour le projet \COREL. En effet, la réadaptation des données demande de repenser les sources numériques. Cette étape est d'autant plus difficile pour le projet \COREL, qui est l'héritier direct de \LSC et \EPJ : les chercheurs ayant travaillé sur les trois projets ont pensé le projet \COREL comme l'aboutissement des deux projets précédents, qui servaient la préparation des données. Dès lors, repenser les données sous une autre forme, à travers le prisme unique du projet \COREL, et non une vision d'ensemble des trois projets, requiert de rompre en partie cette continuité d'un projet à un autre. 

De plus, cette particularité du projet \COREL engendre une réflexion sur le périmètre. Bâtir sur les fondations des projets précédents, en étendant le périmètre pour y intégrer les objectifs du projet actuel, présente le risque de ne pas réussir à produire les livrables du projet. C'est notamment le cas du projet \LSC, qui est actuellement à l'abandon par le propriétaire du site web. Seuls les chercheurs continuent à alimenter ce site, mais ne peuvent qu'ajouter du texte aux codes légaux déjà édités. Les perspectives d'enrichissement du projet \LSC sont ainsi très limitées : comme la maintenance évolutive du site web qui est abandonnée ou la traduction des textes qui est inachevée. 

\subsection{Des données hétérogènes}

Utiliser les productions des projets de recherche antérieurs pose aussi la problématique de l'hétérogénéité des données. En effet, bien que ces trois projets de recherche se veulent complémentaires et continus, leurs données ne sont pas liées entre elles et ne servent pas le même but. Cette vision continue des projets résulte à la fois de l'équipe \COREL, puisque les chercheurs sont familiers des trois projets ; mais aussi de l'objectif du projet, qui regroupe à la fois la volonté d'éditer les textes de lois - l'objectif du projet \LSC - et de retracer la généalogie des lois - l'objectif du projet \EPJ. Le projet \COREL permet ainsi de faire le lien entre les objectifs des deux projets précédents. Toutefois, cela suppose de travailler à partir de sources de données hétérogènes. Lier ces deux sources de données de formats différents (les fichiers \XML et les annotations exportées au format \JSON), est indispensable pour réaliser l'objectif du projet. Ce lien entre dans la chaîne de traitement des données, qui doivent donc être réadaptées et liées entre elles. Bien que cela présente une contrainte supplémentaire, adapter non pas une, mais deux sources de données pour alimenter le projet ne limite pas la préparation des données à une seule solution unique et plusieurs perspectives sont envisageables : il est par exemple possible d'adapter les données en conservant leurs formats respectifs, puis en les liant ; de créer une nouvelle source de données unique, pensée pour le projet \COREL ; de privilégier l'un des deux formats et d'y intégrer les données manquantes. 

\subsection{Un projet de recherche en histoire du droit}

Outre les contraintes liées aux sources de données, le projet présente la contrainte de l'adaptation au public cible du projet. En effet, le projet \COREL est un projet porté par des chercheurs, pour des chercheurs. Dès lors, les humanités numériques sont un moyen de réaliser le projet, mais n'en sont pas la finalité. Il est important de garder ce fil conducteur durant toute la réalisation du projet, car les livrables qui en résultent sont différent selon l'orientation du projet. Il est pertinent de prendre appui sur l'exemple du projet \calendar, qui offre un outil permettant de convertir les dates chinoises (de la dynastie Qing à l'ère républicaine) en dates du calendrier grégorien. Ce projet d'humanités numériques a permis de produire un script python permettant de convertir les dates d'un calendrier à l'autre. Cet outil, produit pour les chercheurs en humanités numériques, n'aurait pas trouvé d'utilité auprès d'un public de chercheurs du droit chinois sans compétences informatiques. Or, l'objectif principal du projet \COREL n'est pas de faire avancer la recherche dans le numérique en créant un programme de reconstitution de la législation chinoise, mais de rendre accessibles et exploitables des sources du droit chinois. Le site internet, mais aussi son back-office, doivent donc être compréhensibles et utilisables par les chercheurs sans compétences techniques. L'utilisation d'outils open-source, en interface graphique, est un aspect essentiel de la bonne réalisation du projet, afin que les chercheurs puissent ensuite développer le site web en autonomie, notamment pour enrichir, modifier ou supprimer son contenu. 

De plus, le projet \COREL est également soumis à des contraintes inhérentes à de nombreux projets de recherche : son financement est limité dans le temps et les ressources budgétaires et humaines sont restreintes. L'équipe est constituée de deux chercheurs et d'un ingénieur qui ne sont pas disponibles à temps plein. Ainsi, tous ces facteurs nécessitent une organisation rigoureuse et une gestion efficace des moyens humains et techniques pour assurer le bon déroulement du projet. 
