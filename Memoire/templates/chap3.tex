\section{Le contexte de travail}
    \subsection{Une équipe issue de la collaboration inter-institutionnelle}

L'équipe du projet \COREL est composée de deux chercheurs : M. Frédéric Constant, chercheur de l'Université Côte d'Azur, responsable scientifique du projet et M. Luca Gabbiani, chercheur de l'\EFEO ; et d'un ingénieur, M. Vincent Paillusson, responsable informatique de l'\EFEO et gestionnaire à la Maison de l'Asie. La responsable administrative du projet, Mme Anne Chatellier, est la directrice des réseaux et partenariats documentaires du \cdf. Le projet s'inscrit ainsi dans un contexte de collaboration inter-institutionnelle et se trouve soumis à des contraintes à la fois géographiques, les membres de l'équipe n'étant pas réunis sur un même lieu de travail, et organisationnelles, puisque l'équipe ne travaille pas à temps plein pour le projet \COREL. 

Rattachée administrativement au \cdf lors du stage, j'ai moi-même été confrontée à ces contraintes depuis un lieu de travail différent des autres membres de l'équipe. Les échanges quotidiens du projet se déroulent majoritairement en distanciel, par mails ou visioconférences et tous les membres de l'équipe n'ont pas toujours la possibilité de se rendre disponible. Des réunions en présentiel, au \cdf ou à l'\EFEO peuvent également être organisées, mais doivent être programmées jusque plusieurs semaines en amont afin d'anticiper les obligations professionnelles de chacun. L'organisation de ces réunions aboutit généralement à des réunions longues, d'une durée supérieure à deux heures, car en plus d'être motivées par un sujet précis, elles représentent des moments rares d'échanges de l'équipe au complet et permettent de problématiser ou de résoudre certaines questions qui dépassent le cadre de la réunion. De plus, les chercheurs sont fréquemment en déplacement professionnel à l'étranger, ce qui demande un niveau d'adaptation supplémentaire : il est nécessaire d'anticiper, en plus des contraintes professionnelles, le décalage horaire. Ces contraintes d'organisation sont exigeantes pour l'équipe et leur demandent parfois de prolonger leurs journées de travail au-delà de leurs horaires habituels, complexifiant davantage les échanges hybrides.  

\subsection{Le travail déjà amorcé}
Dès mon entrée en stage, j'ai été confrontée à ces contraintes d'organisation. J'ai dû faire preuve d'une assimilation rapide des connaissances en découvrant le domaine de l'histoire du droit chinois et comprendre rapidement l'état du projet lors des réunions organisées les premiers jours, qui ont permis de réunir l'équipe du projet mais aussi des chercheurs de ce domaine d'études. Une quantité importante d'informations m'a été donnée sur une période courte de deux jours et j'ai ensuite dû apprendre à travailler à distance avec l'équipe, avec une grande place laissée à l'autonomie.

\subsubsection{Les différentes étapes de travail}
Appréhender les différentes sources, originales ou numériques, a été une difficulté supplémentaire. Le projet dispose d'un corpus assez restreint de cinq textes de lois, mais les données numériques qui en résultent sont multiples et hétérogènes. Il a donc fallu identifier en premier lieu les différentes étapes du travail déjà effectué. Les codes légaux ont d'abord été numérisés par la bibliothèques d'études chinoises puis océrisés. Les données de l'\OCR ont ensuite été encodées en \XML et publiées sur le site \LSC. L'encodage est encore en cours et le texte de l'\OCR est en train d'être corrigé. Les numérisations des codes ont ensuite été confiées au prestataire Data Futures et téléchargées sur un serveur \IIIF. Ces images ont d'abord été segmentées puis annotées. Les annotations sont en cours de corrections et d'enrichissement. 

\subsubsection{Les besoins émergents}
La présentation qui m'a été faite du projet et du travail déjà amorcé a permis aux chercheurs d'évoquer par la même occasion les besoins du projet. 
%difficultés à appréhender les différentes sources différentes et non liées, à démêler le travail des projets précédents VS le travail du projet COREL (qui, au final, n'est qu'en phase d'élaboration ?)
%les besoins qui émergent / un projet un peu à l'arrêt : site web de l'ancien projet qui ne convient plus, le propriétaire ne répond plus, pourtant on continue d'alimenter ces données. Le schéma xml ne convient plus non plus car on veut rajouter du nouveau contenu. 
%pas de pistes pour savoir comment réaliser le projet
%difficultés à définir clairement le périmètre du projet


 \section{Un projet en phase d’initialisation}
    \subsection{Définition de la mission de stage}

§ Paragraphe 1

Idée :\\
Exemple :\\
Référence :\\
Transition :\\

§ Paragraphe 2

Idée :\\
Exemple :\\
Référence :\\
Transition :\\

§ Paragraphe 3

Idée :\\
Exemple :\\
Référence :\\
Transition :\\

\subsection{Les enjeux de la mission}

§ Paragraphe 1

Idée :\\
Exemple :\\
Référence :\\
Transition :\\

§ Paragraphe 2

Idée :\\
Exemple :\\
Référence :\\
Transition :\\

§ Paragraphe 3

Idée :\\
Exemple :\\
Référence :\\
Transition :\\


\section{Les contraintes inhérentes au projet}
    \subsection{S’inscrire dans la continuité de deux projets de recherches}

§ Paragraphe 1

Idée :\\
Exemple :\\
Référence :\\
Transition :\\

§ Paragraphe 2

Idée :\\
Exemple :\\
Référence :\\
Transition :\\

§ Paragraphe 3

Idée :\\
Exemple :\\
Référence :\\
Transition :\\

\subsection{Des données hétérogènes}

§ Paragraphe 1

Idée :\\
Exemple :\\
Référence :\\
Transition :\\

§ Paragraphe 2

Idée :\\
Exemple :\\
Référence :\\
Transition :\\

§ Paragraphe 3

Idée :\\
Exemple :\\
Référence :\\
Transition :\\

\subsection{Un projet de recherche}

§ Paragraphe 1

Idée :\\
Exemple :\\
Référence :\\
Transition :\\

§ Paragraphe 2

Idée :\\
Exemple :\\
Référence :\\
Transition :\\

§ Paragraphe 3

Idée :\\
Exemple :\\
Référence :\\
Transition :\\