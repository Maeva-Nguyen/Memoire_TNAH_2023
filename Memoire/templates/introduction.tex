L'édition scientifique numérique fait l'objet de nombreux projets de recherche. La publication de textes en ligne permet de diffuser des sources, notamment historiques et littéraires, à un plus large public - chercheurs, étudiants ou non-spécialistes. En plus de servir la valorisation des sources, cette diffusion massive de textes en ligne contribue au concept de l'\textit{open data}, une préoccupation importante des humanités numériques : permettre l'accès et la réutilisation des données de la recherche par tous, en respectant les droits d'auteurs. 

Les enjeux de la diffusion des données de la recherche sont aujourd'hui doubles : aux enjeux patrimoniaux s'ajoutent désormais ceux de la science ouverte, introduits par l'usage du numérique au service de la recherche. Plus qu'un outil, l'union du numérique et de la recherche permet à des projets pluridisciplinaires de voir le jour. Ainsi, les chercheurs en histoire agrandissent leur champ de compétences en travaillant conjointement avec des ingénieurs en humanités numériques. 

L'édition scientifique numérique répond à deux besoins majeurs de la recherche : préserver les sources originales tout en valorisant leur contenu. Créer des sources numériques et les publier permet ainsi de démocratiser l'accès aux ressources, disponibles en libre accès. Cependant, cette affluence de données sur le web pose la question des \og bonnes pratiques \fg de la science ouverte. Diffuser librement des données sur le web requiert également de respecter les principes \fair des données pour assurer l'utilité d'une telle entreprise. Pour aider à la mise en place de l'\textit{open data}, le \w indique des standards à respecter afin de garantir l'accessibilité des données sur le web. 

%annonce de problématique/plan
Le projet \COREL est représentatif de ces enjeux pluridisciplinaires : son équipe cherche à assurer la valorisation d'un corpus de textes et l'accès à ces sources via le numérique, en s'inscrivant dans cette démarche de science ouverte. À travers l'étude de ce projet et de ces enjeux, il est pertinent de se demander en quoi l’édition scientifique numérique peut offrir aux chercheurs un accès facilité à des informations dispersées dans différentes sources. Il est également intéressant d'explorer comment le numérique permet de retracer l’évolution de la législation de la Chine impériale année après année et assurer la cohérence d’un code légal reconstitué à partir de sources partielles. Après une présentation du projet et de ses objectifs, l'analyse de l'étape de préparation des données conformément au standard \TEI et la diffusion de ces données sur le web seront le moyen d'évoquer comment un projet de recherche peut contribuer à l'\textit{open data} grâce au numérique.
