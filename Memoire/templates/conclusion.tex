La préparation des données est ainsi une étape non négligeable de l'édition scientifique numérique. Les formats standard des données, comme la \TEI, permettent aux projets de recherche de produire des données \fair et de contribuer à l'ouverture des données de la recherche. La \TEI permet de s'adapter à des corpus très différents grâce à sa nature englobante et son balisage sémantique, ce qui permet de publier des données avec l'appui d'une communauté de chercheurs, dans un langage standard. Les outils tels que \tp permettent aux chercheurs d'accéder plus facilement à la publication de ces données, en interface graphique. De tels outils open source et bien documentés permettent de démocratiser la publication de données sur le web, en rendant accessible aux chercheurs les étapes techniques de la création d'un site web et en facilitant le travail des ingénieurs, ce qui permet une plus étroite collaboration entre les aspects scientifique et technique des projets de recherche. Dans le cadre du projet \COREL, les formats standard de données, dont la mise en place d'une base de données documents aux formats \XML et \JSON, ainsi que l'utilisation d'outils open source contribuent à l'agrégation de sources partielles et rendent ainsi l'accès à ces sources facilité grâce à leur numérisation et leur publication. Cela permet de mettre en place une chaîne de traitement des données afin de retracer l'évolution de la législation de la Chine impériale tardive. Si l'accès à l'édition scientifique numérique se voit démocratisé sur le web, il est toutefois important de garantir la collaboration entre ingénieurs en sciences humaines et chercheurs afin de produire des données interopérables et une plateforme utile aux chercheurs de manière pérenne. 
