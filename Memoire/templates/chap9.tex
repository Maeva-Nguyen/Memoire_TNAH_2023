\section{Le choix de la science ouverte}
    \subsection{Publication des données sous licence libre : la licence Etalab}

Le projet \COREL souhaite produire une édition scientifique numérique \textit{open source} et contribuer à l'enrichissement des données de la recherche. Lors de la rédaction du cahier des charges et de l'établissement d'un jeu de données en \TEI, la question de la licence à utiliser s'est posée. De nombreux projets de recherche utilisent les licences \textit{Creative Commons}, dont le site internet permet de choisir facilement une licence qui correspond aux besoins de chacun. En effet, un questionnaire à remplir permet ensuite de rediriger l'utilisateur vers la licence qui correspond le mieux à ses réponses. Les licences Creative Commons sont également bien documentées afin de permettre à chacun de choisir au mieux la licence qu'il souhaite utiliser. 

Dans un premier temps, la possibilité d'utiliser une licence CC-BY-SA\footnote{https://creativecommons.org/licenses/by-sa/4.0/} a été envisagée par le projet. Cette licence permet d'autoriser la réutilisation des données en attribuant au projet \COREL la paternité des données originales et d'indiquer les modifications effectuées sur les données (BY), et de partager les données dans les mêmes conditions que le projet \COREL, c'est-à-dire en conservant la licence CC-BY-SA. 

Toutefois, le projet \COREL est un projet de recherche public. C'est pourquoi la licence Etalab\footnote{https://www.etalab.gouv.fr/licence-ouverte-open-licence/} a finalement été choisie pour le projet. Mise en place par le gouvernement français dans le cadre de l'\textit{open data}, cette licence est : 

\begin{quote}
    la licence de référence pour les administrations pour la publication de données publiques.\footnote{Ibid.}
\end{quote}

Cette licence est équivalente à la licence CC-BY et est donc compatible avec celle-ci, si le projet \COREL atteint des chercheurs en droit chinois ailleurs que dans le cadre de la réglementation française. Le choix de cette licence s'est imposé afin de respecter la licence mise en place par le gouvernement pour les institutions publiques. La réflexion autour de cette licence a également permis au projet d'envisager des conditions de réutilisation plus libres. En effet, la licence CC-BY-SA présente plus de contraintes que la licence CC-BY ou Etalab, étant donné qu'elle impose aux utilisateurs de repartager les données sous la même licence. La licence CC-BY n'impose aucune restriction et permet de partager ses données en \textit{open access}, sans autre condition que l'attribution de la paternité de l'oeuvre à un tiers. En souhaitant contribuer à l'ouverture des données et au partage des données publiques, le projet \COREL a donc choisi d'utiliser la licence Etalab en France afin de s'inscrire dans une démarche de science ouverte. 

\subsection{Utilisation d’outils open-source, maintenus par une communauté scientifique}

En plus du choix de la licence, il était essentiel pour le projet d'utiliser des outils et langages \textit{open source} et bien documentés, afin d'éviter les écueils des projets précédents. En effet, les données des projets précédents, bien que pensés pour être disponibles en \textit{open access} n'ont en réalité pas contribués à la science ouverte et à l'enrichissement des données de la recherche. Les données du projet \LSC ont été balisées en \XML afin d'utiliser un langage standard de partage des données. Cependant, le manque de documentation et de diffusion de ces données a contribué à l'établissement de données difficiles d'accès et non-réutilisables. Afin de publier des données dans les principes \fair et rétablir l'accès aux sources du droit chinois, le projet \COREL a donc transformé ces données en \TEI, ce qui a permis d'établir un schéma d'encodage bien documenté, qui pourra être consulté par des chercheurs en droit chinois et en humanités numériques. Afin de contribuer à l'ouverture des données, il est également nécessaire de publier ces données afin qu'elles soient accessibles librement sur le web, par exemple sur une page GitHub dédiée au projet.

De plus, la plateforme \tp, maintenue par une communauté scientifique en publication de données \TEI, permet au projet \COREL d'utiliser un outil \textit{open source} et bien documenté. Cela permet d'éviter la création d'une plateforme éphémère comme le site web \LSC, laissé à l'abandon par son propriétaire, sans moyen de l'entretenir. L'usage d'un outil \textit{open source} a pour avantage d'être maintenu par une communauté entière, à l'inverse d'un site propriétaire. De plus, \tp étant fondé sur le \TEI Processing Model, si la plateforme venait à ne plus fonctionner, l'\ODD générée par celle-ci serait toujours utilisable puisqu'elle s'inscrit dans les standards de la \TEI. Néanmoins, l'utilisation d'un outil \textit{open source} n'est pas le seul garant de la maintenabilité du site web du projet dans le temps, même après le financement. Il est essentiel de penser également à cette question de pérennisation des données publiées en ligne, afin de produire une véritable contribution à la science ouverte, et non une plateforme à durée de vie limitée. 

 \section{Perspectives et évolutions du projet }
    \subsection{Maintenance et hébergement}
Afin d'assurer un outil pérenne pour les chercheurs, il est nécessaire de penser en amont à l'hébergement et à la maintenance du site web. Ces deux aspects du projets ont été intégrés au cahier des charges, afin de souligner l'importance d'héberger et de maintenir le site web du projet dans le temps pour ne pas créer une plateforme qui deviendrait obsolète dans quelques années. 

Plusieurs solutions sont envisagées pour l'hébergement. Étant donné que le projet résulte d'une collaboration entre les institutions, il est possible que le \cdf ou l'\EFEO soient, l'un ou l'autre, l'hébergeur du site web. Le projet envisage également de faire appel à Huma-Num pour héberger le site. Huma-Num propose d'héberger gratuitement, cependant la maintenance reste aux frais du projet et doit être garantie afin d'obtenir l'hébergement d'Huma-Num. De plus, d'autres conditions sont spécifiées sur le site web de l'infrastructure\footnote{https://documentation.huma-num.fr/hebergement-web/}, afin d'assurer aux utilisateurs que les données soient ouvertes et interopérables. Les données et métadonnées du site doivent, notamment, être référencées dans Isidore\footnote{Isidore est un moteur de recherche mis en place par Huma-Num pour les sciences humaines et sociales, qui référence des publications scientifiques, colloques et toutes sortes de documents et permet de faire une recherche plein texte dans ces documents.} via le protocole \oai \footnote{Ce protocole permet de garantir l'interopérabilité des données grâce à un standard permettant de diffuser des données et d'en collecter.}. Bien que seul l'hébergement Huma-Num requiert obligatoirement le respect de ce standard, l'interopérabilité et l'ouverture des données sont essentiels pour l'équipe du projet, afin d'enrichir les données de la recherche et de produire un outil accessible aux chercheurs, contrairement au site web précédent qui n'est plus maintenu et ne respecte pas les principes \fair des données. Le référencement dans Isidore fait donc partie des étapes de mise en place du site web, peu importe l'hébergeur choisi. 

Par ailleurs, la maintenance envisagée pour le projet \COREL est essentiellement corrective, afin de garantir un site web pérenne dans le temps. En effet, les chercheurs souhaitent pouvoir mettre à jour le site internet avec de nouveaux documents, sur le même modèle d'encodage défini pour les textes légaux. Si les nouveaux documents \TEI respectent le schéma de validation de l'\ODD, les documents devraient s'afficher correctement grâce à l'\ODD de \tp. Ainsi, le site web du projet ne demande qu'une maintenance corrective, sans ajout de nouvelles fonctionnalités. Cette maintenance sera assurée par Vincent Paillusson et permettra de veiller à la bonne intégration des nouveaux documents sur le site. 

\subsection{Les évolutions envisagées}

L'évolution principale du projet \COREL consiste en l'ajout de nouveaux documents. En effet, des recueils de cas et de jugements sont actuellement en train d'être numérisés par la bibliothèque d'études chinoises. À termes, il est donc envisagé de les intégrer au site internet du projet et de les lier aux lois auxquelles elles sont rattachées. En effet, certains cas donnent naissance à de nouvelles lois ou articles additionnels afin d'adapter la loi à un cas spécifique. Afin d'offrir aux chercheurs un outil permettant d'étudier l'évolution et la généalogie des lois, intégrer ces recueils de cas au site web n'est donc pas sans intérêt pour la recherche. L'ajout de ces nouveaux documents nécessite de les encoder en \TEI sur le même modèle que les textes de lois. Toutefois, le modèle d'encodage ayant été créés pour les codes légaux sans prendre en compte les recueils de cas, le schéma d'encodage devra vraisemblablement être adapté aux nouveaux documents et nécessitera de modifier l'\ODD de \tp ou d'en créer une nouvelle afin de personnaliser l'affichage pour ces documents. Ces évolutions relèvent donc de la maintenance évolutive étant donné que le paramétrage de l'application réalisée avec \tp devra être modifié afin de s'adapter à ces nouveaux ajouts. 

De plus, l'ajout d'autres visualisations ont été évoquées pendant le stage, mais n'ont pas été intégrées au cahier des charges car elles outrepasse le périmètre du projet. Il est toutefois intéressant de conserver ces perspectives d'évolution du projet. Ainsi, les visualisations suivantes peuvent être envisagées comme une évolution du site web du projet : 

\begin{itemize}
    \item Établir une cartographie du code : l'organisation d'un code légal a pu changer dans le temps et certaines lois additionnelles peuvent changer de \lu de rattachement. 
    \item Réaliser des études statistiques sur les textes : le pourcentage de caractères ayant changé d'une version à une autre d'un code légal ou suivre des évolutions de vocabulaire par exemple.
\end{itemize}

Le projet \COREL envisage donc une maintenance corrective afin de maintenir le site web dans le temps et permettre aux chercheurs d'accéder aux données et à l'édition des textes librement. Toutefois, des évolutions plus importantes sont également pensées et nécessiteraient une maintenance évolutive, et probablement un financement supplémentaire afin de les mener à bien. Bien que ces perspectives d'évolution soient hors périmètre dans le cadre du projet, il est important de les envisager en amont afin de fournir aux utilisateurs un site web pérenne et des données utilisables, qui respectent les principes de l'\textit{open data} et soient consultables et réutilisables par les chercheurs et ainsi produire une plateforme utile à la recherche. 

