\section{La collaboration entre les institutions}
\subsection{Le Collège de France}

\subsubsection{Historique}
Le \cdf naît sous le règne de François I\ier{} lorsqu'en 1520, Guillaume Budé, libraire du roi, demande la création d'une institution regroupant des professeurs. C'est ainsi que sont nommés trois professeurs d'hébreux, deux de grec et un de mathématiques en 1530 par François I\ier{} : les lecteurs royaux. La création du \cdf est issue de la croissance de l'humanisme et, en accord avec sa devise \textit{\og Docet omnia \fg}, il enseigne \og tout \fg. C'est d'abord une institution "bâti[e] en hommes", comme le dit Pierre Bayle, qui ne possède pas de siège. Cependant, les cours ont essentiellement lieu place de Cambrai, et un bâtiment y est construit à la fin du XVIII\ieme siècle. Aujourd'hui encore, les cours donnés au \cdf sont publics et ouverts à tous. L'institution s'inscrit ainsi dans une tradition de \og science ouverte \fg. 

\subsubsection{Organisation actuelle}
Le \cdf est divisé en plusieurs chaires. Les chaires statutaires sont attribuées à des professeurs élus, qui occupent cette position jusqu'à l'âge de 73 ans. Les chaires annuelles et internationales sont occupées par des chercheurs choisis par l'Assemblée du \cdf pour des périodes plus courtes. En plus des cours ouverts à tous publics et donnés sur place, le \cdf met à disposition des cours en ligne, des enregistrements, des podcasts, etc. afin de diffuser le savoir plus librement via leur site internet et une chaîne YouTube. 

\subsubsection{La bibliothèque d'études chinoises}
La Bibliothèque d'études chinoise du \cdf a été fondée en 1927 par Paul Pelliot et Marcel Granet. Elle est rattachée à l'\IHEC, lui-même fondé en 1920. La bibliothèque de l'\IHEC rejoint le \cdf en 1972. Elle possède un fonds de 120 000 oeuvres, dont 1 600 périodiques, principalement de la Chine impériale et pré-impériale. Cette collection a reçu le label \CollEx en 2021. La bibliothèque met à la disposition du projet \COREL des sources juridiques et leurs numérisations. Elle est le porteur administratif du projet.

\subsection{L’École Française d'Extrême Orient}

\subsubsection{Historique}
L'\EFEO a été créée en 1898 à Hô Chi Minh-Ville (aussi appelée Saigon) sous le nom de \og Mission archéologique d'Indo-Chine \fg. L'institution naît sous l'influence du courant orientaliste du XIX\ieme siècle en France, afin d'encourager les chercheurs à se rendre en Indo-Chine, et celle du gouvernement d'Indo-Chine, pour préserver le patrimoine indo-chinois. La \og Mission archéologique d'Indo-Chine \fg devient l'\EFEO en 1900.

\subsubsection{Organisation actuelle}
L'\EFEO se situe dans le 16\ieme arrondissement de Paris, au sein de la Maison de l'Asie, un bâtiment partagé avec l'\EPHE et l'\EHESS. L'\EFEO compte dix-huit centres de recherche en Asie, notamment à Pondichéry, Hanoi ou encore Jakarta. Les chercheurs en sciences humaines et sociales de l'\EFEO étudient les civilisations asiatiques à travers un champ pluridisciplinaire (histoire, philologie, sciences de la religion, archéologie...). L'\EFEO collabore avec le \cdf pour le projet \COREL. 

\subsection{Les infrastructures de recherche (l'\IR* Huma-Num et Data Futures)}
\subsubsection{L'\IR* Huma-Num}
L'\IR* Huma-Num est une infrastructure de recherche dans le domaine des humanités numériques et accompagne les chercheurs en sciences humaines. Elle prend appui sur les principes \fair des données et la science ouverte. Son objectif est de permettre aux chercheurs d'accéder librement aux données de la recherche, mais aussi de garantir leur pérennité et leur qualité. 

L'\IR* Huma-Num est partenaire du projet \LSC dont est issu le projet \COREL. 

\subsubsection{Data Futures}
Data Futures est une entreprise à but non-lucratif qui aide à la préservation et à l'accessibilité des données de la recherche. Elle participe à des projets d'humanités numériques à travers les sciences de la vie et les sciences humaines et sociales.

Elle est partenaire du projet \COREL et du projet précédent \EPJ. Le projet fait appel à ses prestations de manière ponctuelle pour le traitement des données.

\section{Présentation du projet \COREL}
\subsection{Description du projet}

§ Paragraphe 1

Idée :\\
Exemple :\\
Référence :\\
Transition :\\

§ Paragraphe 2

Idée :\\
Exemple :\\
Référence :\\
Transition :\\

§ Paragraphe 3

Idée :\\
Exemple :\\
Référence :\\
Transition :\\

\subsection{L’objectif du projet}

§ Paragraphe 1

Idée :\\
Exemple :\\
Référence :\\
Transition :\\

§ Paragraphe 2

Idée :\\
Exemple :\\
Référence :\\
Transition :\\

§ Paragraphe 3

Idée :\\
Exemple :\\
Référence :\\
Transition :\\

\subsection{Les livrables du projet}

§ Paragraphe 1

Idée :\\
Exemple :\\
Référence :\\
Transition :\\

§ Paragraphe 2

Idée :\\
Exemple :\\
Référence :\\
Transition :\\

§ Paragraphe 3

Idée :\\
Exemple :\\
Référence :\\
Transition :\\

\section{Les projets de recherche antérieurs}
\subsection{Le projet Legalizing Space in China}

§ Paragraphe 1

Idée :\\
Exemple :\\
Référence :\\
Transition :\\

§ Paragraphe 2

Idée :\\
Exemple :\\
Référence :\\
Transition :\\

§ Paragraphe 3

Idée :\\
Exemple :\\
Référence :\\
Transition :\\

\subsection{Le projet Emerging Procedural Justice}

§ Paragraphe 1

Idée :\\
Exemple :\\
Référence :\\
Transition :\\

§ Paragraphe 2

Idée :\\
Exemple :\\
Référence :\\
Transition :\\

§ Paragraphe 3

Idée :\\
Exemple :\\
Référence :\\
Transition :\\
