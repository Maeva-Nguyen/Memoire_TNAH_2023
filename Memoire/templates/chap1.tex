\section{La collaboration entre les institutions}
\subsection{Le Collège de France}

\subsubsection{Historique}
Le \cdf naît sous le règne de François I\ier{} lorsqu'en 1520, Guillaume Budé, libraire du roi, demande la création d'une institution regroupant des professeurs. C'est ainsi que sont nommés trois professeurs d'hébreux, deux de grec et un de mathématiques en 1530 par François I\ier{} : les lecteurs royaux. La création du \cdf est issue de la croissance de l'humanisme et, en accord avec sa devise \textit{\og Docet omnia \fg}, il enseigne \og tout \fg. C'est d'abord une institution "bâti[e] en hommes", comme le dit Pierre Bayle, qui ne possède pas de siège. Cependant, les cours ont essentiellement lieu place de Cambrai, et un bâtiment y est construit à la fin du XVIII\ieme siècle. Aujourd'hui encore, les cours donnés au \cdf sont publics et ouverts à tous. L'institution s'inscrit ainsi dans une tradition de \og science ouverte \fg. 

\subsubsection{Organisation actuelle}
Le \cdf est divisé en plusieurs chaires. Les chaires statutaires sont attribuées à des professeurs élus, qui occupent cette position jusqu'à l'âge de 73 ans. Les chaires annuelles et internationales sont occupées par des chercheurs choisis par l'Assemblée du \cdf pour des périodes plus courtes. En plus des cours ouverts à tous publics et donnés sur place, le \cdf met à disposition des cours en ligne, des enregistrements, des podcasts, etc. afin de diffuser le savoir plus librement via leur site internet et une chaîne YouTube. 

\subsubsection{La bibliothèque d'études chinoises}
La Bibliothèque d'études chinoise du \cdf a été fondée en 1927 par Paul Pelliot et Marcel Granet. Elle est rattachée à l'\IHEC, lui-même fondé en 1920. La bibliothèque de l'\IHEC rejoint le \cdf en 1972. Elle possède un fonds de 120 000 oeuvres, dont 1 600 périodiques, principalement de la Chine impériale et pré-impériale. Cette collection a reçu le label \CollEx en 2021. La bibliothèque met à la disposition du projet \COREL des sources juridiques et leurs numérisations. Elle est le porteur administratif du projet.

\subsection{L’École Française d'Extrême-Orient}

\subsubsection{Historique}
L'\EFEO a été créée en 1898 à Hô Chi Minh-Ville (aussi appelée Saigon) sous le nom de \og Mission archéologique d'Indo-Chine \fg. L'institution naît sous l'influence du courant orientaliste du XIX\ieme siècle en France, afin d'encourager les chercheurs à se rendre en Indo-Chine, et celle du gouvernement d'Indo-Chine, pour préserver le patrimoine indo-chinois. La \og Mission archéologique d'Indo-Chine \fg devient l'\EFEO en 1900.

\subsubsection{Organisation actuelle}
L'\EFEO se situe dans le 16\ieme arrondissement de Paris, au sein de la Maison de l'Asie, un bâtiment partagé avec l'\EPHE et l'\EHESS. L'\EFEO compte dix-huit centres de recherche en Asie, notamment à Pondichéry, Hanoi ou encore Jakarta. Les chercheurs en sciences humaines et sociales de l'\EFEO étudient les civilisations asiatiques à travers un champ pluridisciplinaire (histoire, philologie, sciences de la religion, archéologie...). L'\EFEO collabore avec le \cdf pour le projet \COREL. 

\subsection{Les infrastructures de recherche (l'\IR* Huma-Num et Data Futures)}
\subsubsection{L'\IR* Huma-Num}
L'\IR* Huma-Num est une infrastructure de recherche dans le domaine des humanités numériques et accompagne les chercheurs en sciences humaines. Elle prend appui sur les principes \fair des données et la science ouverte. Son objectif est de permettre aux chercheurs d'accéder librement aux données de la recherche, mais aussi de garantir leur pérennité et leur qualité. 

L'\IR* Huma-Num est partenaire du projet \LSC dont est issu le projet \COREL. 

\subsubsection{Data Futures}
Data Futures est une entreprise à but non-lucratif qui aide à la préservation et à l'accessibilité des données de la recherche. Elle participe à des projets d'humanités numériques à travers les sciences de la vie et les sciences humaines et sociales.

Elle est partenaire du projet \COREL et du projet précédent \EPJ. Le projet fait appel à ses prestations de manière ponctuelle pour le traitement des données.

\section{Présentation du projet \COREL}
\subsection{Description du projet}
Le projet \COREL est un projet de recherche en humanités numériques, qui vise à reconstituer l'histoire du droit de la Chine impériale tardive sous la dynastie Qing (1644 - 1911). Ce projet prend appui sur deux projets précédents, les projets \LSC et \EPJ, auxquels le responsable scientifique du projet, M. Frédéric Constant, a également participé. À partir des sources juridiques de la bibliothèque des hautes études chinoises, de leur édition en ligne par le projet \LSC et des numérisations et annotations des sources par le projet \EPJ, le projet \COREL propose à la fois une édition en ligne des textes de lois et une reconstitution de sources partielles grâce au numérique

Le projet a obtenu un financement \CollEx-Persée pour deux ans, jusqu'au mois d'octobre 2024. 

\subsection{L’objectif du projet}
Le projet \COREL a pour objectif de préserver et valoriser un corpus de sources historiques du droit chinois et de les mettre à la disposition des chercheurs, qu'ils soient historiens du droit ou sinologues. Proposer l'agrégation de ces sources partielles et qui se complètent les unes les autres pour faciliter le travail des chercheurs est au c\oe ur de ce projet, c'est pourquoi seuls des outils open-source et bien documentés, consacrés à la publication scientifique numérique et appuyés par une communauté de chercheurs en humanités numériques, sont envisagés pour réaliser les livrables.

\subsection{Les livrables du projet}
Plusieurs livrables sont attendus pour la fin du financement \CollEx-Persée en 2024. Le premier livrable est un site internet qui recrée partiellement celui du projet \LSC en proposant une édition scientifique numérique des codes légaux ainsi que des compilations des codes.
Une fonctionnalité appelée \code sera intégrée au site web afin de reconstituer la législation chinoise entre 1644 et 1911. Ce \code permet de recréer artificiellement la législation à partir l'agrégation des différentes sources du corpus. Pour une année donnée, l'utilisateur pourra consulter toutes les lois en vigueur, ordonnées selon les normes de composition d'un code légal sous la dynastie Qing (divisions en chapitres, lois secondaires classées sous la loi principale à laquelle elles sont rattachées...)

Le projet souhaite aussi intégrer des visualisations permettant de retracer la généalogie d'une loi, sa promulgation, ses modifications et éventuellement ses fusions, divisions ou encore abrogations lorsqu'elles existent. Ces visualisations seront accessibles directement depuis l'édition en ligne, en cliquant sur le titre de la loi dont l'utilisateur souhaite consulter la généalogie. 

Soumis à un financement limité dans le temps, le projet \COREL s'engage auprès de \CollEx-Persée à fournir un \POC de ces livrables, sur un jeu de données réduits si le temps ne permet pas la réalisation de tous ces livrables. Trois lois seront choisies pour représenter le corpus dans son intégralité, avec des cas de figures complexes permettant d'illustrer la faisabilité de la reconstitution de la législation chinoise impériale.


\section{Les projets de recherche antérieurs}
\subsection{Le projet Legalizing space in China}

Le projet \LSC a reçu un financement de l'\ANR de 2011 à 2015. L'objectif de ce projet est de réunir les sources du droit chinois (mais également d'autres pays d'Asie : Mongolie, Corée, Japon, Vietnam) dans une édition en ligne trilingue (chinois, français et anglais). 

Le site web du projet \LSC réunit donc un grand nombre d'éditions en ligne de codes légaux et de compilations, ainsi que leurs numérisations au format PDF. \LSC propose également des tableaux synoptiques des codes, un glossaire qui permet à la fois l'étude des termes juridiques chinois, mais aussi l'harmonisation du travail de traduction, un index des noms de lieux et de personnes et des cartes de la Chine en relation avec les textes de lois. 

Le projet \COREL s'appuie sur une partie des sources utilisées par le projet \LSC et leur édition numérique. Cette édition a été réalisée grâce à des documents encodés en \XML. Ces sources numériques sont reprises dans le cadre du projet \COREL comme socle à la création d'un nouveau jeu de données pour la publication en ligne. 

\subsection{Le projet Emerging procedural justice}
Le projet \EPJ a obtenu un financement Arqus European University Alliance en 2020. Ce projet s'est effectué en collaboration avec la société Data Futures et a permis de déposer sur un serveur \IIIF les numérisations des sources, puis de les annoter et les segmenter pour mettre en avant la structure des textes ainsi que les relations qui existent entre les lois. Ces annotations ont été réalisées à partir d'un fichier de recensement exhaustif des relations entre les lois des différents codes. 

Dans le cadre du projet \COREL, l'export de ces numérisations au format \JSON sont utilisées pour enrichir le jeu de données du projet et retracer la généalogie des lois.  